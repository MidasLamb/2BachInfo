%%%%%%%%%%%%%%%%%%%%%%%%%%%%%%%%%%%%%%%%%%%%%%%%%
%Hoe te gebruiken%%%%%%%%%%%%%%%%%%%%%%%%%%%%%%%%
%%%%%%%%%%%%%%%%%%%%%%%%%%%%%%%%%%%%%%%%%%%%%%%%%
%Gebruik dit .tex bestand als input (\intput{Titlepage.tex}) 
%na begin \begin{document}, in het hoofddocument, gebruik de
%packages: \usepackage{standalone} en \usepackage{framed}
%%%%%%%%%%%%%%%%%%%%%%%%%%%%%%%%%%%%%%%%%%%%%%%%%


\documentclass{article}


\usepackage[paper=a4paper,top=2.85cm, right=2.5cm,left=2.5cm]{geometry}
\usepackage{helvet}

\renewcommand{\maketitle}{\bgroup\setlength{\parindent}{0pt}
\begin{flushleft}
  \textbf{\@title}

  \@author
\end{flushleft}\egroup
}







\begin{document}
\begin{titlepage}
\renewcommand{\familydefault}{\sfdefault}
\normalfont
	\begin{center}
		\begin{framed}
			\LARGE{
			\vspace{0.7cm}
			G0N11C Statistiek \& data-analyse \\
			Project eerste zittijd 2015-2016 
			\vspace{0.8cm}
			}%
		\end{framed}
	\end{center}

\vspace{0.7cm}
\hfill
\\
\large{
\textbf{Naam 1:} Naam1 \hfill \textbf{Studierichting 1:} richting
\\
\hfill
\\
\textbf{Naam 2:} Naam2 \hfill \textbf{Studierichting 2:} richting
\\
\hfill
\\
\textbf{Groepsnummer:} 000
}
\end{titlepage}
%Resets the normal margins after the titlepage.



\end{document}