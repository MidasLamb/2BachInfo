\section{System Protection}
%--------------------------------------------------
\subsection{Goals of Protection}
\begin{itemize}
	\item Kwaadwillige, intentionele schendingen voorkomen.
	\item Elk process mag resources alleen gebruiken zoals vastgelegd door de policies.
\end{itemize}

%---------------------------------------------------
\subsection{Principles of Protection}
\definition{Principle of least privilege}{Maar net genoeg privileges geven om de taak uit te voeren.}
Hier moet men voor iedere user een apparte account hebben met de net die privileges dat die gebruiker nodig heeft.

%-----------------------------------------------------
\subsection{Domain of Protection}
\subsubsection{Domain Structure}
\subsubsection{An Example: UNIX}
\subsubsection{An Example: MULTICS}

%-----------------------------------------------------
\subsection{Access Matrix}

%-----------------------------------------------------
\subsection{Implementation of the Access Matrix}
\subsubsection{Global Table}
\subsubsection{Access Lists for Objects}
\subsubsection{Capability Lits for Domains}
\subsubsection{A Lock-Key Mechanism}
\subsubsection{Comparison}

%-----------------------------------------------------
\subsection{Access Control}


%-----------------------------------------------------
\subsection{Revocation of Access Rights}

%-----------------------------------------------------
\subsection{Capability-Based Systems}
\subsubsection{An Example: Hydra}
\subsubsection{An Example: Cambridge CAP System}

%-----------------------------------------------------
\subsection{Language-Based Protection}
\subsubsection{Compiler-Based Enforcement}
\subsubsection{Protection in Java}

%-----------------------------------------------------
\subsection{Summary}