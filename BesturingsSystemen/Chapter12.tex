\documentclass[12pt]{article}


\newcommand{\definition}[2]{\textbf{#1}: \textit{#2}}

\title{{\Huge Besturingssystemen} \\ {\LARGE Deel 4: Storage management}\\{\Large Hoofdstuk 12: Mass-Storage Structure}}
\date{October 31, 2014}
\author{Midas Lambrichts}

\begin{document}
	\begin{titlepage}
		\maketitle
	\end{titlepage}

%-----------------------------------------------------------------------------------
\section{Overview of Mass-Storage Structure}
\subsection{Magnetic Disks}
\subsubsection{Fysieke eigenschappen}
\definition{Magnetische Schijf}{Een schijf van 1.8-3.5in aan beide kanten bedekt met magnetisch materiaal.} \\
Een `read-write head' vliegt juist over iedere schijf. Deze `head' hangt vast aan een `disk arm'. \\
Het oppervlak van de schijf is logisch ingedeeld in `tracks', die weer onderverdeeld zijn in `sectors'. De set van `tracks' boven/onder elkaar is een `cylinder'. \\
\definition{Transfer rate}{Snelheid van de data tussen de `drive' en de computer}\\
\definition{Positioning time (random-acces time)}{Seek time + rotational latency}\\
\definition{Seek time}{De tijd nodig voor een `disk arm' om boven de juiste `cylinder' te komen}\\
\definition{Rotational latency}{De tijd nodig vooraleer de juiste `sector' onder de `read-write head' komt.} \\
\definition{Head crash}{Normaal zweeft de `read-write head' op een dun luchtkussen boven de disk, als dit niet het geval is kan de `read-write head' contact maken met de disk, dit zal het magnetische oppervlak beschadigen. Dit is een `Head crash'. Dit valt zelden te repareren.} \\ \\
Disks kunnen `removable' zijn. \\ \\
\subsubsection{I/O bus}
\definition{I/O bus}{De set van kabels die de disk drive met de computer verbindt.}
\\ \\
Soorten I/O bus:
\begin{itemize}
	\item \textbf{ATA} advanced technology attachment
	\item \textbf{SATA} serial ATA
	\item \textbf{eSATA}
	\item \textbf{USB} universal serial bus
	\item \textbf{FC} fibre channel
\end{itemize}


\subsection{Solid-State Disks}
\subsection{Magnetic Tapes}
%-----------------------------------------------------------------------------------
\section{Disk Structure}
%-----------------------------------------------------------------------------------
\section{Disk Attachment}
\subsection{Host-Attached Storage}
\subsection{Network-Attached Storage}
\subsection{Storage-Area Network}
%-----------------------------------------------------------------------------------
\section{Disk Scheduling}
\subsection{FCFS Scheduling}

\subsection{SSTF Scheduling}
\subsection{SCAN Scheduling}
\subsection{C-SCAN Scheduling}
\subsection{LOOK Scheduling}
\subsection{Selection of a Disk-Scheduling Algorithm}
%-----------------------------------------------------------------------------------
\section{Disk Management}
\subsection{Disk Formatting}
\subsection{Boot Block}
\subsection{Bad Blocks}
%-----------------------------------------------------------------------------------
\section{Swap-Space Management}
\subsection{Swap-Space Use}
\subsection{Swap-Space Location}
\subsection{Swap-Space Management: An Example}
%-----------------------------------------------------------------------------------
\section{RAID Structure}
\subsection{Improvement of Reliability via Redundancy}
\subsection{Improvement in Performance via Parallelism}
\subsection{RAID Levels}
\subsection{Extensions}
\subsection{Problems with RAID}
%-----------------------------------------------------------------------------------
\section{Stable-Storage Implementation}
%-----------------------------------------------------------------------------------
\section{Summary}

\end{document}