%% Based on a TeXnicCenter-Template by Tino Weinkauf.
%%%%%%%%%%%%%%%%%%%%%%%%%%%%%%%%%%%%%%%%%%%%%%%%%%%%%%%%%%%%%

%%%%%%%%%%%%%%%%%%%%%%%%%%%%%%%%%%%%%%%%%%%%%%%%%%%%%%%%%%%%%
%% HEADER
%%%%%%%%%%%%%%%%%%%%%%%%%%%%%%%%%%%%%%%%%%%%%%%%%%%%%%%%%%%%%
\documentclass[a4paper,twoside,10pt]{report}
% Alternative Options:
%	Paper Size: a4paper / a5paper / b5paper / letterpaper / legalpaper / executivepaper
% Duplex: oneside / twoside
% Base Font Size: 10pt / 11pt / 12pt


%% Language %%%%%%%%%%%%%%%%%%%%%%%%%%%%%%%%%%%%%%%%%%%%%%%%%
\usepackage[USenglish]{babel} %francais, polish, spanish, ...
\usepackage[T1]{fontenc}
\usepackage[ansinew]{inputenc}

\usepackage{lmodern} %Type1-font for non-english texts and characters



%%%%%%%%%%%%%%%%%%%%%%%%%%%%%%%%%%%%%%%%%%%%%%%%%%%%%%%%%%%%%
%% DOCUMENT
%%%%%%%%%%%%%%%%%%%%%%%%%%%%%%%%%%%%%%%%%%%%%%%%%%%%%%%%%%%%%
\begin{document}

\pagestyle{empty} %No headings for the first pages.


%% Title Page %%%%%%%%%%%%%%%%%%%%%%%%%%%%%%%%%%%%%%%%%%%%%%%
%% ==> Write your text here or include other files.

%% The simple version:
\title{Business Information Systems \\ 2015-2016}
\author{Midas Lambrichts}
%\date{} %%If commented, the current date is used.
\maketitle


%% Inhaltsverzeichnis %%%%%%%%%%%%%%%%%%%%%%%%%%%%%%%%%%%%%%%
\tableofcontents %Table of contents
\cleardoublepage %The first chapter should start on an odd page.

\pagestyle{plain} %Now display headings: headings / fancy / ...


\chapter{Business Intelligence and Data Analytics}
\section{Business Intelligence}
\subsection{Intro}
\begin{itemize}
	\item Until 1990, organisations invested in IT to increase efficiency/production.
	\item After 1990, analyzing the data become more important \Rightarrow Transactional systems were not good for this \Rightarrow even more pressure.
\end{itemize}

In Business Intelligence, you need to query a lot of data (which are all transactions) \Rightarrow You need high performance queries \Leftrightarrow Normalised database would require lot of (time consuming) JOINS.
\\
A transactional system is a system which queries 1 record at a time.

\subsection{Business Intelligence Techniques}
\begin{itemize}
	\item \textbf{Verifaction based techniques:} Shows what we already know, does not generate new data. Summarize already existing data. Used for reporting.
	\item \textbf{Disovery based techniques:} Generates new data, shows new things. (e.g. Will people leave the company)
\end{itemize}
\subsubsection{Verification based techniques}
\paragraph{Enterprise reporting} Analytic applications that offer ready-made report templates for industry specific metrics and thresholds for alerts. \\ 
Examples:
\begin{itemize}
	\item Balanced scoreboards
	\item Digital dashboards
	\item Customized reports
\end{itemize}
\subparagraph{Balanced scorecards} 
\begin{itemize}
	\item Strategic level BI technique!
	\item Provides overall view of company \rightarrow See how well company is performing.
\end{itemize}
\subparagraph}Digital dashboards or BI dashboards} 
\begin{itemize}
	\item Visual representation of information.
	\item Tailor-mode for each organisation.
	\item Often browser based.
\end{itemize}

\paragraph{Corporate Performance Management}
\begin{itemize}
	\item Tactical level (Sales, products,...)
	\item Dashboard for operations \rightarrow NOT strategic.
	\item More detailed than balanced scorecards.
	\item Functionally oriented.
\end{itemize}
\subparagraph{Businesss Activity Monitoring}
\begin{itemize}
	\item Real-time reporting.
	\item Summary statistics.
	\item Operational level
\end{itemize}

\paragraph{OLAP}
\begin{itemize}
	\item \textbf{OLAP:} \textbf{O}n\textbf{L}ine \textbf{A}nalytical \textbf{P}rocessing.
\end{itemize}
\subsubsection{Discovery based techniques}


\section{Data Warehousing}


\section{Data Analytics}

\end{document}

