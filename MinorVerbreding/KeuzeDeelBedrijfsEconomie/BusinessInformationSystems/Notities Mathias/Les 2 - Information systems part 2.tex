\documentclass{article}
\setlength\parindent{0pt}

\begin{document}

\section{Business Information systems}

You need three components to create a system (see system les 1), difference with a business information system

\begin{enumerate}
\item  Elements

\item  Relations: collect, search, process, store and distribute information.

\item  Purpose: control decision making processes and coordination.
\end{enumerate}

Input of the system is raw data without context and the system will process this to create relevant information for the business.

\hfill \newline A business information system is a carbon copy of a business system but they don't exist physically anymore, they exist in IT.

\begin{itemize}
\item  The copy of the business system must be perfect or the business information system will fail!
\end{itemize}

\section{Types of information systems}

In many organizations are a lot of business information systems separate from each other. 

\begin{itemize}
\item  Example: Sales and Production
\end{itemize}

\subsection{Management level}

\begin{itemize}
\item  Strategic: Long term decisions

\item  Tactical: Mid long term decisions (more internal)

\item  Operational: Daily operations and decisions with immediate results.
\end{itemize}

\subsection{Operational information systems}

This is typically information from inside the company (see slides for info)

\hfill \newline
Example: Online Transaction system: Monthly pay check for employees

\subsection{Enterprise resource planning systems (ERP)}

For any functional area (Sales, Finance, Human recourses, Production), you will use the same centralised database so the systems are not separated.

\hfill \newline
A lot of companies use the same information to run the company. Plain vanilla ERP systems are such ``off the shelf'' systems that almost every company can use without any configuration specific for the company.

\hfill \newline
You need this type of system to support the next systems which are more advanced (Management systems and executive systems)

\subsection{Information systems on tactical levels}

Tactical decisions that are decided every month, like do we need to change something? Do we need to adept?

\hfill \newline
This information is not always ``clear'', you often need humans to make tactical decisions. It is also possible that you need external information for a tactical decision.

\hfill \newline
Management information system:

\begin{itemize}
\item  A system that covers almost the complete company to make decisions

\item  Outputs reports, charts or a summary which is often used by managers to create a decision or track the decisions that they made before.

\item  The output is often in real-time for quick decision making
\end{itemize}

Decision support system:

\begin{itemize}
\item  A system that covers a specific area of the company

\item  Makes special reports of simulations of what may happen with the current course of business model. These are tailored to a specific system.

\item  Example: How to ship your product to the client? Is it cheaper by plane, boat or maybe truck? The system will recalculate to make adaptions. These are not everyday calculations.
\end{itemize}

\subsection{Strategic information systems}

These systems are difficult to make and are really advances because you need a lot of information.

\hfill \newline
The system needs to process a lot of information and get the information to make a decision to maybe change your production plan (very drastic decision).

\hfill \newline 
Executive support system:

\begin{itemize}
\item  Communication and calculations on strategic levels

\item  Looks like a management information system but is far more advanced because it searches and uses the correct information to make a decision. (a lot of information processing)
\end{itemize}

\subsection{Other systems}

Office Automation Systems:

\begin{itemize}
\item  Is used by almost all people in the company

\item  Office, email client, \dots 
\end{itemize}

Knowledge work systems:

\begin{itemize}
\item  Specialized systems for a specific job
\end{itemize}

\section{Information systems strategy}

\subsection{How are information systems developed?}

\paragraph{Porter priority model:}

\begin{itemize}
\item  Threat of new entrants

\begin{itemize}
\item  How difficult it is to access an industry

\item  Create a new pizza company: fairly easy

\item  Create a new chip company: almost impossible because of a lot of research, knowledge and money
\end{itemize}

\item  Bargaining power of suppliers

\begin{itemize}
\item  Many types of companies with a lot of competition vs monopoly

\item  Consequence: different prices and quality

\item  The company needs to have a system that is profitable.
\end{itemize}

\item  Bargaining power of buyers

\begin{itemize}
\item  How sensitive costumers are to buy a product

\item  If costumers are willing to pay a high price, the company can sell their product for a high price (example: Apple)
\end{itemize}

\item  Threat of substitute products or services

\begin{itemize}
\item  New services may cause a huge problem for current organisation because the new service is a lot more efficient.

\item  Example: Airlines vs collaborative video meeting solutions

\item  Example: Taxi companies vs Uber
\end{itemize}

\item  Rivalry among existing competitors

\begin{itemize}
\item  How do you compete with the current competitors?

\item  Can the organisations avoid competing about the price, if this is the case it may destroy the industry because the prices aren't profitable anymore.

\item  Counter-logical strategy: Colruyt =$>$ lowest price, by saying this they avoid a price wars because they say that they will always have the lowest price.
\end{itemize}
\end{itemize}

\paragraph{Porter's competitive forces theory: Summary}

\begin{itemize}
\item   + : Generic, applicable to any company in any industry

\item   + : Provides detailed explanations for firm/industry profitability

\item   + : Reason about an organisation's strategic position with respect to its external environment

\item   -- : Of limited use for determining future strategy

\item   -- : Few concrete guidelines on how to operationalize
\end{itemize}

\section{Competitive advantage and information systems}

\subsection{Value chain model}

Sequence of activities through which inputs are transformed into more valuable outputs.

\hfill \newline
This value chain doesn't exist on its own, you don't only need to look internally also externally. You need a value chain system which brings together the value chain of the suppliers, customers and your own. And also the value chains of your competitors. 

\hfill \newline
A value chain on its own is useless and thus cannot be used to create efficient decisions.

\hfill \newline
Value Chain + Information system

\begin{itemize}
\item  Create and sustain your current business model without losing to the competition.
\end{itemize}

\end{document}

