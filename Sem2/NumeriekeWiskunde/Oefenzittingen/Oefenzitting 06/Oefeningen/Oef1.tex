\documentclass[../Oefenzitting4.tex]{subfiles}
\begin{document}
  \subsection{Oefening 1:}
    Beschouw de tweede-graadsveelterm $f(x) = 2x^{2} + 4x -5$
    \begin{itemize}
      \item a) Bereken de interpolerende veelterm $y_{1}(x)$ in de punten $\{-1, 1 \} $ met de methode van Lagrange en met de methode van Newton.
      \\
      $
        x_{0} = 1, f_{0} = f(1) = 1
        \\
        x_{1} = -1, f_{1} = f(-1) = -7
        \\
        l_{0} = \frac{x+1}{1-(-1)} = \frac{1}{2}x + \frac{1}{2}
        \\
        l_{1} = \frac{x-1}{-1-1} = \frac{-1}{2}x + \frac{1}{2}
        \\
        y_{1} = \frac{1}{2}x + \frac{1}{2} - 7(\frac{-1}{2}x \frac{1}{2})
        \\
        = 4x -3
      $
      \\
      Methode van newton:
      \\
      $
        \begin{matrix}
          1 & 1  \\
          1 & -7
        \end{matrix}
        \\
        \Rightarrow \frac{-7 - 1}{-1 - 1} = 4
        \\
        y_{1} = 1+4(x-1) = 4x - 3
      $

      \item b) Stel het Vandermonde stelsel op en ga na dat je oplossing hieraan voldoet.
      \\
      $
      \begin{bmatrix}
        1 & 1  \\
        1 & -1
      \end{bmatrix}
      \begin{bmatrix}
        -3 \\
        4
      \end{bmatrix}
      =
      \begin{bmatrix}
        1 \\
        -7
      \end{bmatrix}
      $

      \item c) Bereken de interpolatiefout $E_{1}(x)$ ...
      \\
      $
          E_{1}(x) = 2x^{2} + 4x -5 - (4x -3)
          \\
          = 2x^{2} - 2
      $
    \end{itemize}
\end{document}
