\documentclass[../Oefenzitting4.tex]{subfiles}
\begin{document}
  \subsection{Oefening 2:}
    Beschouw dezelfde tweede-graadsveelterm $f(x) = 2x^{2} + 4x -5$ en herhaal probleem 1 met de interpolatiepunten $ \{ -1,0,1 \} $ waarbij je voor de methode van Newton de twee soorten tabellen van gedeelde differenties opstelt. Wat is de waarde van $ E_{2}(x)$? Verklaar.
    \begin{itemize}
    \item a) Bereken de interpolerende veelterm $y_{1}(x)$ in de punten $\{-1, 1 \} $ met de methode van Lagrange en met de methode van Newton.
    \\
    $
      x_{0} = -1, f_{0} = f(1) = -7
      \\
      x_{1} = 0, f_{1} = f(0) = -5
      \\
      x_{2} = 1, f_{2} = f(1) = 1
      \\
      l_{0} = \frac{(x-0)(x-1)}{(-1-0)(-1-1)} = \frac{1}{2}x(x-1)
      \\
      l_{1} = ... = 1-x^{2}
      \\
      l_{2} = ... = \frac{1}{2}x(x+1)
      \\
      y_{2} = \frac{-7}{2}x(x-1) - 5(1-x^{2}) + frac{1}{2}x(x+1)
      \\
      = 2x^{2} + 4x - 5
    $
    \\
    Methode van newton:
    \\
    $
      \begin{matrix}
        -1 & -7  \\
        0  & -5  \\
        1  & 1
      \end{matrix}
      \\
      \Rightarrow
      \\ \frac{-5+7}{0+1} = 2
      \\ \frac{1+5}{1-0} = 6
      \\
      \Rightarrow
      \\ \frac{6-2}{1+1} = 2
      \\
      y_{2} = -7 +2(x+1) + 2(x+1)x = 2x^{2} + 4x - 5

    $

    \item b) Stel het Vandermonde stelsel op en ga na dat je oplossing hieraan voldoet.
    \\
    $
    \begin{bmatrix}
      1 & 1  \\
      1 & -1
    \end{bmatrix}
    \begin{bmatrix}
      -3 \\
      4
    \end{bmatrix}
    =
    \begin{bmatrix}
      1 \\
      -7
    \end{bmatrix}
    $

    \item c) Bereken de interpolatiefout $E_{1}(x)$ ...
    \\
    $
        E_{1}(x) = 2x^{2} + 4x -5 - (4x -3)
        \\
        = 2x^{2} - 2
    $
  \end{itemize}


\end{document}
