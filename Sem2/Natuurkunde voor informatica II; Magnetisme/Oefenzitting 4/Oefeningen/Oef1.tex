\documentclass[../Oefenzitting4.tex]{subfiles}
\begin{document}
  \subsection{Oefening 1:}
    Twee delen: de buitenste cirkel en de binneste cirkel, want het punt ligt in de verlenging van de rechte stukken, dus die dragen niets bij.
    \\
    $
      B = \int dB = \int k_{m} \frac{I d\vec{s} \times \hat{r}}{r^{2}}
    $
    \\
    $\Downarrow$ $B_{1}$: $r=b$ en $I$ is constant.
    \\
    $
      B_{1} = \int dB = \int k_{m} \frac{I d\vec{s} \times \hat{b}}{b^{2}}
      \\
      B_{1} = k_{m}\frac{I}{b^{2}} \int d\vec{s}
      \\
      B_{1} = k_{m}\frac{I}{b^{2}} \int_0^{\frac{1}{3}\pi b} d\vec{s}
      \\
      = \frac{\mu_{0}I}{4\pi b^{2}} \frac{1}{3} \pi b
      \\
      = \frac{\mu_{0}I}{4b} \frac{1}{3}
      \\
      = \frac{\mu_{0}I}{12b}
    $
    \\
    Volledig analoog voor $B_{2}$
    \\
    $
      B_{2} = \frac{\mu_{0}I}{12a}
    $
    \\
    Totaal magnetisch veld:
    \\
    $
      \hat{B}_{T} = \hat{B}_{2} - \hat{B}_{1}
      \\
      = \frac{\mu_{0}Ib - \mu_{0}Ia}{8ab}
      \\
      = \frac{\mu_{0}I(b-a)}{8ab}
      \\
    $
\end{document}
