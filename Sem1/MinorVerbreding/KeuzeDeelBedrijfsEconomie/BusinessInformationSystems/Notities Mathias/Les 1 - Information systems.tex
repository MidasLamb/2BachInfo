\documentclass{article}
\setlength\parindent{0pt}

\begin{document}

\section{Introduction}

The Challenges that today's companies are confronted with due to the technology explosion:

\begin{enumerate}
\item  Many new Competitors

\begin{itemize}
\renewcommand{\labelitemi}{$\Rightarrow$}
\item  Everyone has their own systems and strategies
\end{itemize}

\item  Who is responsible in the new business firm?

\begin{itemize}
\renewcommand{\labelitemi}{$\Rightarrow$}
\item  Decisions: IT and CEO not only the IT
\end{itemize}

\item  You need people that are capable with IT and the technological challenges that come with it. You can't start a new `successful' company without IT. 

\begin{itemize}
\renewcommand{\labelitemi}{$\Rightarrow$}
\item  People that know and can use the new technological changes to keep a business running.
\end{itemize}
\end{enumerate}

What are the things companies need to do to hit their digital sweet spot?

\begin{enumerate}
\item  They need to have priorities

\item  You will have to make decisions where the technological challenges will create new opportunities

\item  Where and how to invest in IT, how to spread this? Do you have the skills to realize this?

\item  Create an end-to-end view with the costumers: Don't only think about the strategies, also think about the costumers. Otherwise your business model will fail!
\end{enumerate}

If you as a company try to go digital, only a third of all the companies who try this will succeed!

\begin{itemize}
\item  This is because most of this project are created by IT people who don't really use a business model that is correct for their project.

\item  Only 8\% of the IT budget create new successful projects, this is a third of the new project. 75\% of the IT-budget is needed to maintain your current IT-frame;
\end{itemize}

The problem: The gap between IT and business is too big with a lot of new IT projects!

\begin{itemize}
\item  You need planning, resources and user involvement

\item  A project doesn't immediately work at start-up (unrealistic expectations).
\end{itemize}

\section{How big is the impact of IT?}

\subsection{Macro-economic impact of IT:}

The world became flat (See Toledo TheWorldIsFlat -- ThomasLFriedman)

\begin{itemize}
\item  Communication between two places on earth is immediate
\end{itemize}

\subsection{Micro-economic impact of IT:}

IT doesn't matter (See Toledo ITDoesntMatter -- NichalosGCarr)

\begin{itemize}
\item  Follow IT advancement don't lead or you have a bigger chance to fail

\item  Invest only if others have succeeded!
\end{itemize}

Counter-perspective on IT doesn't matter

\begin{itemize}
\item  The value of the IT is not in the technology itself but in the reasoning between your business model and your costumers to create a good business.

\item  The article of Nichalos Carr was written after the `dotcom bubble' where a lot of IT businesses went bankrupt.

\item  If you look at the current businesses we do find that IT is productive

\item  It is not only about automating your company with IT, you also need to rethink about your business model with recent IT technologies.

\item  IT doesn't only automate your company, but it can process all the new information a lot faster and efficient.
\end{itemize}

\section{What is an information system (IS)?}

Data vs Information

\begin{itemize}
\item  Data is raw observed facts without a real context/meaning

\item  Information is knowledge with a context that you can understand!
\end{itemize}

Metadata

\begin{itemize}
\item  Making information more comprehensible to humans and computers

\item  Example: xml files
\end{itemize}

We need data/information but we also need a system

\begin{itemize}
\item  You need elements, relations between these elements and a purpose for the system to work
\end{itemize}

\end{document}

