\documentclass[]{report}

\begin{document}
  \section{UML}
    \begin{itemize}
      \item \textbf{Complete}: A superclass is abstract, all instances are one of the subclasses $\leftrightarrow$ \textbf{Incomplete}
      \item \textbf{Disjoint}: No object can be in two subclasses at the same time $\leftrightarrow$ \textbf{Overlap}
    \end{itemize}

  \section{Logical Data Modeling}
    \begin{itemize}
      \item \textbf{Candidate key}: A minimal (no attribute can be removed) determinant (uniquely identifies) of a relation.
      \item \textbf{Primary key}: An assigned Candidate key.
      \item \textbf{Alternate key}: All Candidate keys except for the primary key.
      \item \textbf{Foreign key}: Set of attributes in a relation that refers to a Primary key in another relation.
    \end{itemize}

    \subsection{Normalisation}
      \begin{itemize}
        \item \textbf{First Normal Form}
        \begin{itemize}
          \item Each attribute is atomic
          \item Each attribute contains 1 value (so no 2 locations in location column)
        \end{itemize}
        \item Second Normal Form
        \begin{itemize}
          \item In First Normal Form
          \item Full functional dependency from the key (If only dependent on part of key, split away)
        \end{itemize}
        \item \textbf{Third Normal Form}
        \begin{itemize}
          \item In Second Normal Form
          \item No transitive dependencies (Professor(\textbf{EmpID}, EmpName, \textbf{Faculty}, \textbf{FacultyWebSite}))
        \end{itemize}
      \end{itemize}

\chapter{BI and Data Analytics}
  \section{Business Intelligence}
    \begin{itemize}
      \item BI: Interactive process for exploring, analyzing, and reporting data. Deriving insights and drawing conclusions.
      \item Developed to support strategical and tactical decisions and to asses business performance.
      \item Verification based and Discovery based techniques.
    \end{itemize}

  \section{Data Warehousing}
    \begin{itemize}
      \item OLTP (On-line Transaction Processing)
      \begin{itemize}
        \item Normalization!
        \item Fast concurrent data acces
        \item Poor performance with lots of joins
      \end{itemize}
      \item OLAP (On-line Analytical Processing)
      \begin{itemize}
        \item Focused on analytical queries.
        \item No normalization.
      \end{itemize}
      \item Data Warehouse:
      \begin{itemize}
        \item Data integration: 80\% of building effort.
        \item Time variant: snapshots
        \item Non-volatile
      \end{itemize}
      \item Logical Data warehouse design
      \begin{itemize}
        \item Relational OLAP (ROLAP)
        \begin{itemize}
          \item Star schema: Facts table (normalized) + Dimension tables (Not normalized)
          \item Snowflake schema: Star schema with normalized dimension tables.
        \end{itemize}
        \item Multidimensionsal OLAP (MOLAP)
        \item Hybrid OLAP (HOLAP)
        \item OLAP operations:
        \begin{itemize}
          \item Roll up: Aggregate dimensions into fewer (Cities $\rightarrow$ Countries)
          \item Drill down: Reverse of Roll up
          \item Slice: For only 1 item of a certain dimensions, removes a dimension (Look at the results where City= Paris)
          \item Dice: Look at results for a subset. (City = Paris or Lyon and Quarter = Q1 or Q3)
        \end{itemize}
      \end{itemize}
      \item Data Staging:
      \begin{itemize}
        \item Data cleaning (KUL, KU Leuven,...)
        \item Data transformation
        \item Load
        \item Refresh
      \end{itemize}
      \item Data mart: Supports decisions of a specific group and is a part of the warehouse.
      \item Data warehouse usage:
      \begin{itemize}
        \item Verification and Discovery
      \end{itemize}

    \end{itemize}


  \section{Data Analytics}
    \subsection{Big Data}
      \begin{itemize}
        \item Huge: Shitton of data
        \item Everywhere: Data is acquired everywhere
        \item Job oppertunities are sky-rocketing: High demand for Big Data Analysis skills
        \item Scary:
        \item Promising: New insights and discoveries
      \end{itemize}

    \subsection{Knowledge Discovery in Databases (KDD)}
      \begin{itemize}
        \item Process of automatically identifying models and patterns from massive observational databases that are: Valid (Holds on new data as well), Novel (Not obvious), Useful, Understandable.
        \item Data Pre-processing:
        \begin{itemize}
          \item How to deal with missing values?
          \item What to do with Outliers?
          \item Defintion of target variable? (What is a "churner"?)
        \end{itemize}
        \item Data mining tasks:
        \begin{itemize}
          \item Predictive methods (supervised)
          \begin{itemize}
            \item Learn based on labeled data (Good-Bad is known)
          \end{itemize}
          \item Descriptive methods (unsupervised)
        \end{itemize}
      \end{itemize}
    \subsection{Predictive Data mining}



\chapter{Powerpoint exercises}
  \begin{itemize}
    \item PPT 5B slide 41: Impurities
  \end{itemize}



D&M

  System Quality
  Information Quality
  Service Quality
  Intention to use
  User Satisfaction
  Net Benefits


TAM
  Perceived usefulness
  Perceived ease of use
  Attitude
  Behavioral intentions
  actual use

\end{document}
