\documentclass{report}
\begin{document}

\chapter{Introduction}
\textbf{operating system} = programma =$>$ beheert de hardware van de computer
\\*= basis voor application programma's
\\*= intermediary tussen de gebruiker en de hardware
\\* =$>$ verschillende soorten operating systems:
\begin{itemize}
	\item Mainframe 
	\\*=$>$ optimaal gebruik van hardware
	\item Personal computer(\textbf{PC})
	\\*=$>$ ondersteunen zowel complexe games als business 		applications en alles ertussen
	\item Mobile computers
	\\*=$>$ omgeving met gemakkelijke interface om programma's uit te voeren
\end{itemize}
DUS: sommige operating systems zijn ontworpen om gemakkelijk te zijn, andere om effici\"ent te zijn en andere een combinatie van de twee.

\section{What Operating Systems Do}
computersysteem:
\begin {itemize}
	\item \textbf{hardware}
	\\*= \textbf{central processing unit(CPU)}, \textbf{memory} en \textbf{input/output (I/O) devices}
	\\* =$>$ basic computing resources voor systeem
	\item operating system
	\\* =$>$ controleert de hardware en co\"ordineert het gebruik over de verschillende application programs voor de verschillende gebruikers
	\item \textbf{application programs}
	\\*=$>$ hoe de resources gebruikt worden om de user' computing problems oplossen
	\item users
\end {itemize}
computersysteem ook als: hardware, software en data
\\*operating system =$>$ geen nuttige functie op zich
\\*= omgeving waarin andere programma's nuttig kunnen werken

\subsection{User View}
\begin{itemize}
	\item PC
	\\*=$>$ bestaat uit: monitor, toetsenbord, muis en system unit
	\\*=$>$ ontworpen zodat \'e\'en gebruiker alle resources krijgt om het werk van de gebruiker te maximaliseren
	\\*= ontworpen voor \textbf{ease of use}(=gebruiksgemak) met een beetje aandacht voor performance en geen voor \textbf{resource utilization}(=hoe divers hardware en software resources verdeeld zijn)
	\item \textbf{mainframe} of \textbf{minicomputer}
	\\*=$>$ bestaat uit: terminals verbonden met dezelfde mainframe of minicomputer
	\\*=$>$ ontworpen voor maximale resource utilization
	\item \textbf{workstations}
	\\*=$>$ bestaat uit: workstations verbonden met netwerken van andere workstations en \textbf{servers}
	\\*=$>$ ontworpen als compromis tussen individueel gebruik en resource utilization
	\item mobile computers
	\\*= standalone units voor individuele gebruikers vaak verbonden met netwerken door cellular of andere wireless technologies
	\\*=$>$ user interface gebruikmakend van \textbf{touch screen} door drukken en vegen met vingers over het scherm
	\item embedded computers
	\\*= computers met weinig of geen user view
	\\*=$>$ ontworpen om zonder user intervention te runnen
\end{itemize}

\subsection{System View}
operating system:
\begin{itemize}
	\item\textbf{resource allocator}
	\\*= kent de resources toe bij vele en mogelijk conflicting resource requests zodat het computersysteem effici\"ent en eerlijk werkt
	\item\textbf{control program} 
	\\*=$>$ beheert de uitvoering van user programma's om errors te voorkomen en onjuist gebruik van de computer
\end{itemize}

\subsection{Defining Operating Systems}
in 1960: \textbf{Moore's Law} voorspelde dat het aantal transistors op een integrated circuit zou elke 18 maanden verdubbelen
\\=$>$ computers werden meer functioneel en kleiner 
\\=$>$ meer toepassingen en een groot aantal diverse operating systems
\\
\\fundamentele doel van computersystemen = gebruikersprogramma's uitvoeren en het oplossen van gebruikersproblemen gemakkelijker maken
\\ =$>$ hardware en software ontwikkeld voor dit doel
\\
\\operating system = het ene programma dat altijd runt op de computer = \textbf{kernel}
\\=$>$ ook nog: 
\begin{itemize}
\item \textbf{system programs}
\\= geassocieerd met operating system maar niet noodzakelijk deel van de kernel
\item application programs
\\= alle programma's die niet geassocieerd worden met operating system
\end{itemize}
mobile operating systems: vaak niet alleen core kernel maar ook \textbf{middleware}(= set of software frameworks die aanvullende services voorzien voor application developers)
\\bv.: iOS en Android =$>$ middleware die databases, multimedia en graphics ondersteunt

\section{Computer-System Organization}
\subsection{Computer-System Operation}
modern general-purpose computersysteem
\\=$>$ bestaat uit: \'e\'en of meer CPU's en een aantal device controllers(\'e\'en voor elk specifiek type device) verbonden met een gemeenschappelijke bus die toegang voorziet tot shared memory
\\=$>$ CPU en device controllers kunnen in parallel uitvoeren met behulp van de memory controller die de toegang tot het shared memory synchroniseerd
\\
\\computer =$>$ te beginnen runnen (bij opstarten of reboot) =$>$ intieel programma 
\\= \textbf{bootstrap program}
\\=$>$ bewaard in hardware in \textbf{read-only memory(ROM)} of \textbf{electrically erasable programmable read-only memory(EEPROM)} = \textbf{firmware}
\\=$>$ initialiseert het hele systeem
\\=$>$ moet weten hoe het operating system te laden en uit te voeren
\\= operating-system kernel localiseren en in memory laden
\\
\\sommige services =$>$ buiten de kernel door system programs die bij boot time in memory zijn geladen
\\ = \textbf{system processes} of \textbf{system daemons}
\\ =$>$ runnen zolang de kernel runt
\\
\\optreden van een gebeurtenis =$>$ gesignaliseerd door \textbf{interrupt} van:
\begin{itemize}
\item hardware
\\=$>$ op elk moment door signaal te sturen naar de CPU, meestal met de system bus
\item software
\\=$>$ door een speciale operatie (=\textbf{system call}= \textbf{monitor call}) uit te voeren
\end{itemize}
wanneer CPU onderbroken =$>$ stopt wat het bezig is en verplaatst onmiddelijk uitvoering naar vaste locatie(die het startadres van de service routine voor de interrupt bevat)
\\=$>$ interrupt service routine wordt uitgevoerd
\\=$>$ CPU hevat de onderbroken berekening
\\
\\interrupts = belangrijk deel van computerarchitectuur
\\=$>$ elk computer design =$>$ eigen interrupt mechanism, maar verschillende functies zijn algemeen
\\interrupt =$>$ controle overdragen aan de gepaste interrupt service routine
\\ =$>$ straightforward method: algemene routine oproepen om de interrupt information te onderzoeken =$>$ routine roept de interrupt-specific handler op
\\ MAAR: interrupts moeten snel behandeld worden
\\=$>$ maar aantal voorgedefineerde interrupts mogelijk dus table met pointers naar interrupt routines
\\=$>$ interrupt routine wordt indirect opgeroepen door de table zonder intermediate routine
\\=$>$ in algemeen: table met pointers bewaard in low memory(houden de adressen bij van de interrupt service routines voor de verschillende devices) = array of \textbf{interrupt vector} van adressen wordt ge\"indexeerd door een uniek device number meegegeven met de interrupt request
\\
\\adres van onderbroken instructie =$>$ ook bewaard worden
\\=$>$ oude designs =$>$ vaste locatie
\\=$>$ recente designs =$>$ op de system stack

\subsection{Storage Structure}
CPU kan enkel instructies uit het geheugen laden
\\=$>$ general-purpose computers runnen meeste programma's vanuit rewritable memory = main memory(\textbf{random-access memory(RAM)})
\\=$>$ main memory =$>$ ge\"implementeerd in semiconductor technology = \textbf{dynamic random-access memory(DRAM)}
\\andere vormen van memory:
\begin{itemize}
\item ROM
\\ = niet veranderbaar, alleen statische programma's
\\bv.: bootstrap program
\item EEPROM
\\ = veranderbaar maar niet vaak, voornamelijk statische programma's
\\bv.: smartphones =$>$ om fabrieksinstellingen te herstellen
\end{itemize}
elke vorm memory =$>$ array van bytes(met elke byte eigen adres)
\\=$>$ explicit loads en stores
\\=$>$ CPU laadt ook automatisch instructies van main memory voor uitvoering
\\
\\ systeem met \textbf{von Neumann architecture}
\\=$>$ haalt eerst een instructie uit memory en bewaart die in \textbf{instruction register}
\\=$>$ instructie wordt vertaald en operands kunnen uit memory gehaald worden en in een intern register bewaard worden
\\=$>$ instructie wordt op operands uitgevoerd
\\=$>$ resultaat wordt terug in memory bewaard
\\memory unit =$>$ alleen een serie memory adressen(niet hoe gegenereerd of waarvoor)
\\
\\ideaal: programma's en data permanent in main memory
\\ =$>$ niet mogelijk:
\begin{itemize}
\item te klein
\\ om alle benodigde programma's en data permanent te bewaren
\item \textbf{volatile}(=vluchtig)storage device
\\=$>$ verliest inhoud zonder stroom of die op een andere manier verloren is gegaan
\end{itemize}
DUS: \textbf{second storage} als uitbreiding van main memory met als hoofdvereiste het permanent bewaren van grote hoeveelheden data
\\ =$>$ meest voorkomende: \textbf{magnetic disk}(storage voor zowel programma's als data)
\\
\\storage structuur met registers, main memory en magnetic disk = maar \'e\'en van de vele mogelijke
\\=$>$ andere met: cache memory, CD-ROM, magnetic tapes
\\=$>$ verscheidenheid kan georganiseerd worden in hierarchy(hogere levels = duur en snel, lager gaan: kost per bit daalt en acces time neemt toe)
\\=$>$ registers, cache, main memory en solid-state disk = semiconductor memory
\\verschillende storage systems:
\begin{itemize}
\item \textbf{volatile storage}
\\= registers, cache, main memory
\\= verliest inhoud zonder stroom
\item \textbf{nonvolatile storage}
\\= solid-state disk, magnetic disk, optical disk, magnetic tapes
\\=$>$ zonder een dure batterij en generator backup system => data op nonvolatile storage om veilig te bewaren
\end{itemize}
\textbf{solid-state disks}
\\ = sneller dan magnetic disks en nonvolatile
\\ verschillende soorten:
\begin{itemize}
\item data bewaren in grote DRAM array
\\=$>$ bevat ook een verborgen magnetic hard disk en een batterij voor backup power
\item flash memory
\\ =$>$ in camera's, \textbf{personal digital assistents(PDA's)},robots en storage op general-purpose computers
\\=$>$ trager dan DRAM maar geen stroom nodig om inhoud te behouden
\item \textbf{NVRAM}
\\= DRAM met batterij backup power
\\= even snel als DRAM en zolang de batterij het uithoudt novolatile
\end{itemize}

\subsection{I/O Structure}
afhankelijk van de device controller kan meer dan \'e\'en device vastgehecht worden
\\bv.: \textbf{small computer-systems interface(SCSI)}controller =$>$ 7 of meer devices
\\
\\device controller
\\=$>$ bestaat uit: wat local buffer storage en een set van special-purpose registers
\\=$>$ verplaatst de data tussen de randapparaten dat het controleert en zijn local buffer storage
\\=$>$ voor elke device controller =$>$ \textbf{device driver}(begrijpt de device controller en voorziet de rest van het operating system van een uniforme interface naar het device toe
\\
\\start I/O operation
\begin{enumerate}
\item device driver: laadt de gepaste registers in de device controller
\item device controller: inhoud bekijken =$>$ welke actie 
\item device controller: data verplaatsen van device naar locale buffer
\item device controller: verplaatsen van data = compleet =$>$ interrupt naar device driver dat operatie gedaan is
\item device driver: geeft de controle terug aan het operating system 
\\=$>$ bij read: geeft data of pointer naar data terug
\\=$>$ bij andere operatie: geeft status informatie terug
\end{enumerate}
=$>$ \'e\'en interrupt per byte voor low-speed devices
\\= vorm van interrupt-driven I/O 
\\=$>$ goed voor kleine hoeveelheden data
\\=$>$ te veel overhead bij heel veel data verplaatsing (bv.: disk I/O)
\\ oplossing: \textbf{direct memory access(DMA)}
\\=$>$ na opzetten buffers, pointers en counters voor I/O device =$>$ device controller verplaatst een hele blok data direct naar of van zijn buffer storage naar het memory zonder tussenkomst van CPU
\\=$>$ maar \'e\'en interrupt per blok dat operatie gedaan is
\\=$>$ CPU beschikbaar voor andere taken
\\sommige high-ends systems eerder switch dan bus architecture

\section{Computer-System Architecture}
\subsection{Single-Processor Systems}
single-processor system:
\\=\'e\'en main CPU in staat om general-purpose instruction set(ook instructies van gebruikersprocessen)uit te voeren
\\=$>$ bevat ook andere special-purpose processors: device-specific processors(bv.: disk, keyboard en graphics controllers) of op mainframes =$>$ meer general-purpose processors
\\
\\special-purpose processors 
\\=$>$ run limited instruction set en runnen geen user processes
\\=$>$ soms beheerd door operating system: stuurt informatie over volgende taak en controleert hun status
\\=$>$ in andere systemen of omstandigheden: special-purpose microprocessors = low-level components gebouwd in de hardware, operating system kan er niet mee communiceren, voeren taken autonoom uit
\\
\\ maar \'e\'en general-purpose CPU =$>$ single-processor system (aantal special-purpose microprocessoren maakt niet uit)

\subsection{Multiprocessor Systems}
\textbf{multiprocessor systems}= \textbf{parallel systems} =\textbf{multicore systems}
\\= 2 of meer processors die communiceren, de computer bus delen en soms de clock, memory en randapparaten
\\=$>$ eerst in servers, dan in desktop en laptop systemen en vervolgens op mobile devices(zoals smartphone en tablet computers)
\\=$>$ 3 voornaamste voordelen:
\begin{enumerate}
\item increased throughput
\\=$>$ meer werk gedaan in minder tijd
\\=$>$ speed-up ratio met N processors is minder dan N
\item economy of scale
\\=$>$ multiprocessor systemen kosten minder dan equivalente meerdere single-processor systemen =$>$ randapparaten, massa-opslag en stroomtoevoer gedeeld 
\item increased reliability
\\=$>$ als functies goed verdeeld over verschillende processors =$>$ falen van \'e\'en processor stopt het systeem niet =$>$ alleen trager
\\=$>$ cruciaal in vele applicaties
\end{enumerate}
\textbf{graceful degradation}
\\= vermogen om service te voorzien proportioneel aan het aantal overlevende hardware
\\=$>$ verder dan graceful degradation 
\\= \textbf{fault tolerant}
\\=$>$ elke single component kan falen en systeem kan toch verder operatie uitvoeren
\\=$>$ mechanisme om falen te detecteren, diagnoseren en indien mogelijk corrigeren
\\bv.: HP NonStop(voordien Tandem) =$>$ meerdere paren CPU's die in lockstep werken =$>$ resultaat van de 2 processoren niet gelijk =$>$ proces verplaatst naar ander paar processor 
\\ =$>$ duur want speciale hardware en aanzienlijk wat hardware duplication
\\
\\ multiple-processor systems:
\begin{itemize}
\item \textbf{asymmetric multiprocessing}
\\=$>$ elke processor is een specifieke taak toegekend
\\=$>$ boss processor die werk schedult en toekent aan worker processors
\item \textbf{symmetric multiprocessing(SMP)}
\\=$>$ in meeste systemen
\\=$>$ alle processors zijn gelijken
\\=$>$ elke processor =$>$ eigen set registers en local cache, maar delen physical memory
\\bv. AIX
\end{itemize}
verschil tussen symmetric en asymmetric multiprocessing =$>$ resulteert van ofwel hardware of software
\\=$>$ special hardware om meerdere processors te onderscheiden of software die maar \'e\'en baas toelaat en meerdere werkers
\\
\\multiprocessing =$>$ voegt CPU's toe om de computing power te doen toenemen
\\=$>$ als CPU een integrated memory controller heeft =$>$ toevoegen van CPU's =$>$ toename van memory addressable in het systeem 
\\=$>$ kan ervoor zorgen dat een systeem zijn memory acces model verandert van \textbf{uniform memory access(UMA)} naar \textbf{non-uniform memory access(NUMA)}
\begin{itemize}
\item UMA
\\= toegang tot elke RAM vanaf elke CPU duurt even lang
\item NUMA
\\= bij sommige delen van memory duurt het langer om te accessen dan andere delen
\\=$>$ operating systems kunnen NUMA penalty minimaliseren door resource management
\end{itemize}
recent in CPU design: meerdere computing cores op single chip
\\= \textbf{multicore} multiprocessor systems
\\= effici\"enter dan dan meerdere chips met single cores =$>$ on-chip communication is sneller dan between-chip communication
\\=$>$ ook minder stroomverbruik bij \'e\'en chip met meerdere cores
\\
\\\textbf{blade servers}
\\=$>$ meerdere processor boards, I/O boards en networking boards worden in dezelfde chassis geplaatst
\\=$>$ verschil met traditionele multiprocessor systems = elke blade-processor board boots onafhankelijk en runt zijn eigen operating system 

\subsection{Clustered Systems}
\textbf{clustered system}
\\= type van multiprocessor system dat meerdere CPU's samen verzameld
\\=$>$ verschillen van multiprocessor systems: samengesteld uit 2 of meer individuele systemen(of nodes) die samengevoegd zijn
\\=\textbf{loosely coupled}
\\=$>$ elke node =$>$ single processor system of multicore system
\\in algemeen: clustered computers delen storage en zijn closely gelinkt via een local-area network LAN of een snellere interconnect(bv.: InfiniBand)
\\
\\ clustering =$>$ om \textbf{high-availability} service te voorzien
\\ =$>$ service gaat door zelfs als \'e\'en of meer systemen in de cluster falen
\\ in algemeen high-availability bekomen door: een niveau van redundancy(overtolligheid)in het systeem toe te voegen
\\ =$>$ laag van van cluster software runt op de cluster nodes
\\=$>$ elke node kan \'e\'en of meerdere van de andere nodes controleren (over de LAN)
\\=$>$ wanneer gecontroleerde machine faalt =$>$ controlemachine neemt storage over en herstart de applicaties die aan het runnen waren op de gefaalde machine
\\=$>$ gebruikers zien enkel korte onderbreking van service
\\
\\clustering:
\begin{itemize}
\item \textbf{asymmetric clustering}
\\=\'e\'en machine in \textbf{hot-standby mode} terwijl de andere de applicaties runt
\\=$>$ hot-standby host machine controleert de actieve server en zodra die server failt, wordt de hot-standby host de actieve server
\item\textbf{symmetric clustering}
\\= twee of meer hosts runnen applicaties en controleren elkaar
\\=$>$ effici\"enter =$>$ gebruikt alle beschikbare hardware
\\=$>$ vereist wel meer dan \'e\'en applicatie beschikbaar om te runnen
\end{itemize}
cluster bestaat uit verschillende computersystemen verbonden via een netwerk
\\=$>$ gebruikt worden om \textbf{high-performance computing} environments te voorzien
\\=$>$ kunnen aanzienlijk grotere computational power leveren dan single-processor of SMP systems omdat een applicatie tegelijk kan runnen op alle computers in de cluster
\\=$>$ applicatie geschreven in voordeel van cluster = \textbf{parallelization}
\\=$>$ verdeelt een programma in afzonderlijke componenten die in parallel runnen op individuele computers in de cluster
\\
\\andere vormen van clusters: parallel clusters en clustering over een wide-area network(WAN)
\\parallel clusters =$>$ meerdere hosts om dezelfde data te accessen in shared storage
\\=$>$ gebruik van speciale versies van software en speciale releases van applicaties is vereist (bv.: Oracle Real Application Cluster = speciale versie van Oracle's database)
\\=$>$ elke machine runt Oracle en een laag software volgt de toegang tot de shared disk
\\=$>$ elke machine =$>$ volledige toegang tot alle data in database
\\=$>$ shared acces =$>$ acces control en locking om conflicting operations te voorkomen door het syteem = \textbf{distributed lock manager(DLM)} 
\\
\\cluster technology verandert snel =$>$ veel verbeteringen mogelijk door \textbf{storage-area networks(SAN's)} 
\\=$>$ staat vele systemen toe om vast te maken aan een pool of storage
\\=$>$ applications en data bewaard op SAN =$>$ cluster software kan de applicatie toekennen om te runnen op elke hosted vastgehecht aan de SAN
\\=$>$ als host faalt =$>$ andere host kan overnemen
\\in database cluster: heel veel hosts kunnen zelfde database delen
\\=$>$ performance en reliability nemen toe

\section{Operating-System Structure}
single program =$>$ meestal niet of CPU of I/O devices hele tijd bezighouden
\\single user =$>$ vaak meerdere programma's running
\\=$>$\textbf{multiprogramming} =$>$ toenemende CPU utilization door jobs te organizeren zodat CPU altijd \'e\'en om uit te voeren
\\=$>$ operating system =$>$ verschillende jobs tegelijk in memory
\\=$>$ main memory te klein voor alle jobs, dus initieel op disk bewaard in \textbf{job pool}(= alle processen die op de disk verblijven in afwachting van toekenning aan main memory)
\\=$>$ wanneer uitvoerend process moet wachten =$>$ ander proces uitvoeren uit job pool
\\
\\logische uitbreiding multiprogramming = \textbf{timesharing} of \textbf{multitasking}
\\= CPU voert meerdere jobs uit door ertussen te switchen maar zo frequent dat de gebruikers met elk programma kunnen interacten terwijl het runt
\\=$>$ vereist \textbf{interactive} computersysteem = voorziet directe communicatie tussen user en systeem
\\=$>$ user geeft instructies aan operating system of programma direct door toetsenbord, muis, touch pad of touch screen en wacht voor onmiddelijke resultaten op output device
\\=$>$ \textbf{response time} moet kort zijn, minder dan 1 seconde
\\
\\time-shared operating system =$>$ meerdere gebruikers op computer tegelijkertijd
\\=$>$ indruk dat computer systeem toegewijd is aan \'e\'en gebruiker hoewel het verdeeld wordt over vele gebruikers
\\
\\\textbf{process} = programma dat in het geheugen geladen wordt en uitvoert
\\time sharing en multiprogramming =$>$ vereisen verschillende jobs tegelijk in geheugen 
\\=$>$ verschillende jobs klaar om in geheugen gebracht te worden =$>$ niet genoeg plek voor allemaal =$>$ systeem moet kiezen met behulp van \textbf{job scheduling}
\\\textbf{CPU scheduling} = kiezen welke job eerst runt 
\\
\\in time sharing systeem =$>$ operating system moet redelijke response time verzekeren =$>$ soms door \textbf{swapping}(= waarbij processen in en uit het hoofdgeheugen naar disk worden verwisseld
\\meer gebruikte methode voor redelijke response time = \textbf{virtual memory}(= laat uitvoering van proces toe dat niet helemaal in geheugen zit)
\\=$>$ grote voordeel: laat gebruikers toe om grotere programma's te runnen dan eigenlijke \textbf{physical memory}
\\=$>$ hoofdgeheugen wordt weergegeven door grote uniforme arrray van storage, waarin \textbf{logical memory} wordt gescheiden van physical memory
\\
\\time-sharing systeem =$>$ moet ook file systeem, mechanisme om resources te beschermen van ongepast gebruik, mechanisme voor job sychronizatie en communicatie voorzien

\section{Operating-System Operations}
moderne operating systems = \textbf{interrupt driven}(= als er niets te doen is, wacht operating system tot er iets gebeurt)
\\gebeurtenissen worden gesignaliseerd door \textbf{trap} of \textbf{exception} = software-generated interrupt ofwel door error ofwel door specifiek request van user program dat operating-system service nodig heeft
\\voor elke interrupt =$>$ aparte segments of code bepalen welke actie 
\\=$>$ interrupt service routine voorzien om interrupt af te handelen
\\
\\omdat operating system en users hardware en software resources delen =$>$ zorgen dat error in user programma alleen problemen veroorzaakt voor het ene programma dat runt
\\=$>$ met sharing =$>$ vele processen nadelig be\"invloed door bug in \'e\'en programma 
\\ deftig ontworpen operating system moet verzekeren dat incorrect programma er niet voor kan zorgen dat andere programma's incorrect uitvoeren
\subsection{Dual-Mode And Multimode Operation}
onderscheiden van uitvoering:
\begin{itemize}
\item operating-system code
\item user-defined code
\end{itemize}
meeste besturingssystemen voorzien hardware-ondersteuning die toelaat om af te wisselen tussen verschillende modes van uitvoering
\\=$>$ tenminste 2 modes nodig:
\begin{itemize}
\item \textbf{user mode}
\item \textbf{kernel mode}
\\= \textbf{supervisor mode}
\\= \textbf{system mode}
\\= \textbf{privileged mode}
\end{itemize}
bit om huidige mode aan te geven in hardware = \textbf{mode bit}
\begin{itemize}
\item kernel
\\ = 0
\item user
\\ = 1
\end{itemize}
bij system boot time hardware =$>$ in kernel mode
\\met dual mode =$>$ operating systeem beschermen tegen errant users(en errant users tegen elkaar)
\\bescherming door sommige machine-instructies die schade kunnen veroorzaken aan te duiden als \textbf{privileged instructions}(= alleen in kernel mode uitgevoerd)
\\
\\concept van modes =$>$ uitgebreid met meer modes dan twee
\\=$>$ CPU's die virtualizatie ondersteunen =$>$ vaak aparte mode om aan te geven wanneer \textbf{virtual machine manager(VMM)} (en virtualization management software) controle heeft over systeem
\\=$>$ meer privileges dan user processen maar minder dan kernel
dus: eerst in kernel mode instructies uitvoeren =$>$ dan controle aan user applicatie =$>$ mode op user mode gezet =$>$ daarna controle terug aan operating system via interrupt, trap of system call
\\
\\system calls = methode gebruikt door proces om actie aan te vragen bij operating system
\\=$>$ wordt behandeld door hardware als software interrupt
\\
\\gebrek aan hardware-supported dual mode =$>$ serieuze tekortkomingen in systeem veroorzaken
\\hardware protection detecteert errors die modes schenden =$>$ normaal afgehandeld door operating system

\subsection{Timer}
om te vermijden dat user programma in oneindige loop terechtkomt om of faalt om system services te callen en controle nooit teruggeeft aan operating system
\\=$>$ \textbf{timer}(= gezet worden om computer te interrupten na specifieke periode))
\\=$>$ periode =$>$ vast of variabel
\\textbf{variable timer} =$>$ meestal ge\"implementeerd door fixed-rate klok en counter
\\voordat controle aan user wordt gegeven =$>$ zeker zijn dat timer gezet is om te onderbreken
\\timer gebruiken om te voorkomen dat user programma te lang runt

\section{Process Management}
programma doet niets totdat instructies worden uitgevoerd door CPU
\\programma in uitvoering = proces
\\proces =$>$ bepaalde resources nodig
\\programma = passief, proces = actief
\\single-threaded proces =$>$ \'e\'en \textbf{program counter} dat volgende instructie die moet uitgevoerd worden specifieert
\\=$>$ uitvoering ervan =$>$ sequentieel 
\\proces = unit van work in system
\\system =$>$ bestaat uit verzameling processen:
\begin{itemize}
\item operating-system processen
\\=$>$ voeren system code uit
\item user processen
\\=$>$ voeren user code uit
\end{itemize}
alle deze processen =$>$ gelijktijdig uitgevoerd worden
\\operating system =$>$ volgende activiteiten in verband met process management:
\begin{itemize}
\item schedulen processen en threads op CPU's
\item creëren en verwijderen van user en system processen
\item opschorten en hervatten processen
\item mechanismen voor process synchronizatie voorzien
\item mechanismen voor process communication voorzien
\end{itemize}

\section{Memory Management}


\chapter{System Structures}
\section{Operating-System Services}
operating system services voor gebruiksgemak van programmer
\\=$>$ functies die nuttig zijn voor gebruiker:
\begin{itemize}
\item user interface
\\ =$>$ bijna alle operating systems =$>$\textbf{user interface(UI)}
\\in verschillende vormen:
\begin{itemize}
\item \textbf{command-line interface}
\\= toetsenbord om in commands te typen in een specifiek format met specifieke opties
\item \textbf{batch interface}
\\= waarin commands en directives om de commands te controleren in files worden ingegeven =$>$ files uitgevoerd
\item \textbf{graphical user interface(GUI)}
\\= window system met pointing device voor direct I/O, kiezen uit menu's en selecties maken en toetsenbord om tekst in te voeren
\end{itemize}
sommige systemen voorzien twee of alle drie
\item program execution
\\systeem =$>$ programma in memory laden en uitvoeren
\\programma =$>$ uitvoering be\"eindigen ofwel normaal ofwel abnormaal(indicating error)
\item I/O operations
\\=$>$ running programma kan I/O vereisen =$>$ file of I/O device
\\=$>$ specific devicces =$>$ speciale functies
\\=$>$ voor efficiency en protection =$>$ gebruikers kunnen I/O devices niet direct controleren =$>$ operating system moet I/O regelen
\item file-system manipulation
\\=$>$ programma's =$>$ read, write, create, delete op naam, zoeken naar gegeven file, file informatie oplijsten =$>$ files en directories
\\=$>$ permissions management om toegang tot file toe te staan of weigeren op basis van file ownership 
\\vele operating systems =$>$ verschillende file systems
\item communications
\\=$>$ communicatie tussen processen: op dezelfde computer of op hetzelfde netwerk
\\=$>$ ge\"implementeerd via:
\begin{itemize}
\item \textbf{shared memory}
\\ = 2 of meer processen lezen en schrijven in shared section of memory
\item \textbf{message passing}
\\= packets information in voorgedefineerde formats worden verplaatst tussen processen door het operating system
\end{itemize}
\item error detection
\\ operating system =$>$ voortdurend errors detecteren en corrigeren
\\ voor elk type error =$>$ gepaste actie om correct en consistent computing te verzekeren
\end{itemize}
=$>$ functies om de effici\"ente operation van het systeem zelf te verzekeren
\\=$>$ systemen met meerdere gebruikers =$>$ effici\"enter door computer resources 
te delen tussen gebruikers
\begin{itemize}
\item resource allocation
\\=$>$ meerdere gebruikers of meerdere jobs tegelijkertijd =$>$ resources allocated aan elk van hen
\\=$>$ routines voor CPU te schedulen, toekennen van printers, USB storage drives en andere randapparaten
\item accounting
\\= usage statistics(= welke gebruikers hoeveel en welke computer resources gebruiken)
\item protection en security
\begin{itemize}
\item protection
\\= alle toegang tot system resources wordt gecontroleerd
\item security
\\= systeem beveiligen tegen outsiders meestal door user authentication
\end{itemize}
\end{itemize}

\section{User and Operating-System Interface}
verschillende manieren voor gebruiker to interface met operating system: command-line interface(= \textbf{command interpreter}, laat gebruikers toe om direct commands in te geven voor operating system) en graphical user interface(GUI)
\subsection{Command Interpreters}
command interpreter =$>$ in kernel of als speciaal programma behandeld(Window en UNIX)
\\op systemen met meerdere command interpreters = \textbf{shells}(bv.: op UNIX en Linux Bourne shell, C shell, Bourne-Again shell, Korn shell,..)
\\=$>$ hoofdfunctie = volgende user-specified command krijgen en uitvoeren(vooral manipuleren files)
\\=$>$ op 2 manieren ge\"implementeerd:
\begin{enumerate}
\item command interpreter zelf bevat code om het command uit te voeren
\item meeste commands worden door system programs ge\"implementeerd 
\\=$>$ command interpreter verstaat commando niet
\\=$>$ commando wordt gebruikt om file met die naam in het geheugen te laten en uit te voeren =$>$ gemakkelijk om nieuwe commands te cre\"eren door nieuwe file met gepaste naam
\\=$>$ commando-interpreter program moet niet veranderd worden voor nieuwe commands en kan klein zijn
\end{enumerate}
\subsection{Graphical User Interfaces}
= user-friendly graphical user interface of GUI
\\=$>$ mouse-based window-and-menu system gekenmerkt door bureaublad(= \textbf{desktop})
\\=$>$ gebruiker verplaatst wijzer van de muis naar afbeeldingen(=\textbf{icons}) op het scherm(desktop) die programma's, files,directories en system functions voorstellen
\\=$>$ afhankelijk van de locatie van de wijzer van de muis =$>$ klikken op een knop:
\begin{itemize}
\item programma oproepen, file of direcory(=\textbf{folder} selecteren
\item menu laten zien dat commands bevat
\end{itemize}
=$>$ first GUI = Xerox Alto computer in 1973
\\=$>$ Apple Macintosh computers in 1980s =$>$ Aqua met Mac OS X
\\
\\ muis =$>$ onpraktisch voor meeste mobile systems =$>$ interactie door \textbf{gestures} op touchscreen(bv.: drukken en vegen)
\\ in plaats van fysiek toetsenbord =$>$ simulatie toetsenbord op touchscreen
\\
\\UNIX systems =$>$ vooral command-line interfaces
\\maar ook GUI beschikbaar: Common Desktop Environment CDE, X-Windows systems en open-soure(K Desktop Environment(KDE)en GNOME, source code hiervan kan dadelijk gelezen en aangepast worden 

\subsection{Choice of Interface}
= keuze van persoonlijke voorkeur 
\\ \textbf{system administrators}(beheren computers) en \textbf{power users}(die uitgebreide kennis hebben van systeem) =$>$ command-line interface, want: snellere toegang tot activiteiten die zij willen uitvoeren
\\op sommige systemen: alleen subset van system functies beschikbaar via GUI
\\command-line interfaces = eigen programmeerbaarheid =$>$ herhalende taken gemakkelijker
\\ \textbf{shell scripts} = programma's met taken die ge\"interpreteerd worden door de command-line interface (wordt niet gecompiled in uitvoerbare code)
\\ =$>$ vooral op systemen die command-line georiënteerd zijn = UNIX, Linux 
\\ Windows: vooral gebruik van Windows GUI en bijna nooit MS-DOS shell interface
\\ Mac OS: had eerst geen command-line interface, nu Aqua interface(GUI) en command-line interface

\section{System Calls}
\textbf{system calls} voorzien interface voor de services die beschikbaar worden gemaakt door een operating system 
\\=$>$ beschikbaar als routines geschreven in C en C++ (buiten zekere low-level taken in assembly-language instructies)
\\ voorbeeld om gebruik van system calls te illustreren: programma schrijven om data van een file te lezen en te kopi\"eren naar een andere file
\begin{enumerate}
\item programma heeft input nodig
\\= naam van 2 files: input file en output file
\begin{itemize}
\item in interactive system:
\\=$>$ reeks system calls (prompting message op scherm schrijven en van toetsenbord de karakters van de file names lezen)
\item op mouse-based en icon-based system:
\\=$>$ menu met file names in een window =$>$ source naam selecteren =$>$ window kan geopend worden voor de destination name te specifieren
\\=$>$ veel I/O system calls
\end{itemize}
\item programma moet input file openenen en output file cre\"eren
\\=$>$ \'e\'en system call per operatie
\\=$>$ mogelijke error conditions =$>$ vereisen bijkomende system calls
\\=$>$ bij openen input file: geen file met die naam of geen toegangsrechten tot die file =$>$ programma print message(reeks system calls) en dan abnormaal be\"eindigen(system call)
\\=$>$ bij cre\"eren output file: al een outputfile met die naam
\begin {itemize}
\item 
\end {itemize}

\end{enumerate}


\chapter{Process Concept}
batch system =$>$ voert \textbf{jobs} uit
\\time-shared systeem =$>$ \textbf{user programs} of \textbf{tasks}
\\zelfs single-user system =$>$ verschillende programma's tegelijk = \textbf{processes}

\section{Process Concept}
\subsection{The Process}
proces  = programma in uitvoering
\\=$>$ meer dan programma code = \textbf{text section}
\\=$>$ bevat ook huidige activiteit 
\begin{itemize}
\item waarde van \textbf{program counter}
\item inhoud registers
\item \textbf{stack}
\\ = bevat tijdelijke data, bv.: functieparameters, returnadressen)
\item \textbf{data section}
\\ = bevat globale variabelen
\item \textbf{heap}
\\ = geheugen dat dynamisch wordt toegewezen tijdens proces runtime
\end{itemize}
programma = passieve entity zoals \textbf{executable file} (= file dat lijst instructies bevat bewaard op disk)
\\proces = actieve entity
\\programma wordt proces wanneer uitvoerbare file in geheugen wordt geladen door:
\begin{itemize}
\item double-clicking icon 
\item entering naam in command line
\end{itemize}
proces zelf kan execution environment zijn voor andere code (bv.: Java programming environment)

\subsection{Process State}
staat van proces =$>$ bepaald door huidige activiteit van proces
\\mogelijke staten:
\begin{itemize}
\item new
\item running
\item waiting
\item ready
\item terminated
\end{itemize}

\subsection{Process Control Block}
elk proces wordt in operating system voorgesteld door \textbf{process control block(PCB)} = \textbf{task control blok}
\\=$>$ bevat veel stukken informatie geassocieerd met specifiek proces:
\begin{itemize}
\item process state
\item program counter
\item CPU registers
\item CPU-scheduling information
\item memory-management information
\item accounting information
\item I/O status information
\end{itemize}

\subsection{Threads}
proces = programma dat single \textbf{thread} van executie uitvoert
\\=$>$ meeste operating systems hebben process concept uitgebreid om toe te laten dat een proces meerdere uitvoeringsthreads heeft en dus meer dan \'e\'en taak kan uitvoeren tegelijk

\section{Process Scheduling}
doel van mulitprogramming: aantal processen altijd hebben runnen om CPU utilization te maximaliseren
\\doel van time sharing: CPU zo vaak tussen processen switchen dat gebruikers kunnen interacten met elk programma terwijl het runt
\\om deze doelen te bereiken: \textbf{process scheduler} selecteert beschikbaar proces voor program execution op CPU

\subsection{Scheduling Queues}
%verder pagina 109
\chapter{Multithreaded Programming}
\chapter{Process Scheduling}
\section{Basic Concepts}
multiprogramming =$>$ sommige processen die de hele tijd runnen om CPU gebruik te maximaliseren
\\=$>$ verschillende processen tegelijk in geheugen
\\=$>$ wanneer proces moet wachten wordt de CPU aan ander proces gegeven
\\dit soort scheduling = fundamentele operating-system functie
\subsection{CPU-I/O burst}
succes van CPU scheduling hangt af van eigenschap van processen
\\=$>$ process uitvoering bestaat uit \textbf{cycle}(= processen wisselen af tussen de twee staten) van CPU execution en I/O wait
\\procesuitvoering begint met \textbf{CPU burst}
\\vervolgens \textbf{I/O burst}, dan weer CPU burst, weer I/O burst en eindigen met CPU burst
\\I/O-bound programma =$>$ veel korte CPU bursts
\\CPU-bound programma =$>$ paar lange CPU bursts
\\=$>$ verdeling kan belangrijk zijn in selectie van gepaste CPU-scheduling

\subsection{CPU scheduler}
wanneer CPU = vrij =$>$ operating system =$>$ proces uit ready queue selecteren om uitgevoerd te worden
\\selectieproces door \textbf{short-term sheduler}(= CPU scheduler)
\\= selecteert proces uit geheugen die klaar zijn om uit te voeren en kent CPU toe aan dat process
\\=$>$ records in ready que =$>$ meestal process control blocks(PCBs) van processen

\subsection{Preemptive Scheduling}
CPU-scheduling beslissingen onder volgende 4 omstandigheden:

\begin{enumerate}
\item switch van running state naar waiting state van proces
\item switch van running state naar ready state van proces
\item switch van waiting state naar ready state van proces
\item wanneer proces eindigt
\end{enumerate}

scheduling alleen onder omstandigheden 1 en 4 =$>$ scheduling schema = \textbf{nonpreemtive} of \textbf{cooperative}
\\=$>$ wanneer CPU toegekent is aan proces =$>$ CPU pas vrijgelaten ofwel door be\"eindiging proces ofwel door switchen naar waiting state
\\=$>$ vereist geen speciale hardware
\\
\\anders: \textbf{preemptive} scheduling
\\=$>$ kan resulteren in race conditions wanneer data gedeeld worden over verscheidene processen
\\=$>$ be\"invloedt ontwerp van operating-system kernel
\\=$>$ sectie met code be\"invloedt door interrupts moet beschermd worden tegen simultaan gebruik
\\=$>$ operating system moet bijna altijd interrupt accepteren
\\=$>$ anders: input verloren of output overschreven

\subsection{Dispatcher}
andere component betrokken bij CPU-scheduling functie = \textbf{dispatcher}(= module dat controle geeft van CPU aan proces dat geselecteerd is door short-term scheduler
=$>$ bestaat uit volgende:
\begin{itemize}
\item switching context
\item switching naar user mode
\item jumping naar juiste locatie in user program om programma te herstarten
\end{itemize}
=$>$ moet zo snel mogelijk zijn
\\\textbf{dispatch latency}= tijd die nodig is voor dispatcher om \'e\'en proces te stoppen en ander te beginnen runnen

\section{Scheduling Criteria}
\begin{itemize}
\item CPU utilization
\\=$>$ CPU zo veel mogelijk bezig houden
\\=$>$ conceptueel CPU gebruik: van 0 tot 100 procent
\\=$>$ in echt systeem: van 40 procent(= licht geladen systeem) tot 90 procent(= zwaar beladen)
\item throughput
\\\textbf{throughput} = aantal processen die compleet zijn per tijdseenheid
\\=$>$ lange processen: \'e\'en proces per uur
\\=$>$ korte processen: tien processen per seconde
\item turnaround time
\\= interval vanaf toekenning van proces tot wanneer proces compleet is 
\\= som van de periodes van wachten om in geheugen te geraken, wachten in ready queue, uitvoeren op CPU en I/O doen
\item waiting time
\\= som van de periodes al wachtend gespendeerd in ready queue
\item response time
\\= tijd vanaf toekenning of request totdat eerste antwoord wordt geproduceerd (= tijd dat het vraagt om te beginnen antwoorden, niet tijd dat vraagt om antwoord te outputten)
\\in algemeen: gelimiteerd door snelheid output device
\end{itemize}
=$>$ CPU utilization en throughput maximaliseren 
\\=$>$ turnaround time, waiting time en response time minimaliseren

\section{Scheduling Algorithms}
\subsection{First-Come, First-Served Scheduling}
\textbf{first-come, first served(FCFS)}= simpelste scheduling algoritme
\\=$>$ proces dat eerst CPU aanvraagt, krijgt eerst CPU toegekent
\\=$>$ ge\"implementeerd met FIFO-queue
\\werking: wanneer proces in ready queue binnenkomt, PCB wordt gelinkt aan tail queue
\\wanneer CPU is vrij, wordt toegewezen aan process aan head queue
\\=$>$ running proces wordt verwijderd uit queue
\\
\\positief: code voor FCFS scheduling =$>$ gemakkelijk te schrijven en begrijpen
\\negatief: average waiting time onder FCFS = nogal lang
\\
\\\textbf{Gantt chart} = bar chart dat particular schedule illustreert inclusief start en finish tijden van elk van de deelnemende processen
\\
\\\textbf{convoy effect} = alle andere processen wachten voor \'e\'en groot proces om af te CPU te gaan 
\\=$>$ resulteert in lagere CPU en device utilization dan mogelijk kan zijn als de kortere processen eerst zouden gaan
\\FCFS = nonpreemptive

\subsection{Shortest-Job-First Scheduling}
\textbf{shortest-job-first(SJF)}
\\=$>$ associeert met elk proces de lengte van de volgende CPU burst van proces
\\=$>$ wanneer CPU beschikbaar =$>$ toegekend aan proces met de kleinste volgende CPU burst
\\als de volgende CPU bursts van twee processen hetzelfdee zijn =$>$ FCFS scheduling beslist welke eerst
\\meer gepaste naam = shortest-next-CPU-burst
\\SJF scheduling algorithm = bewijsbaar optimaal
\\=$>$ geeft minimale gemiddelde wachttijd voor gegeven set processen 
\\moeilijkheid SJF: lengte kennen van volgende CPU aanvraag
\\SJF =$>$ vaak gebruikt in long-term scheduling, kan niet ge\"implementeerd worden bij short-term CPU scheduling
\\
\\volgende CPU burst =$>$ meestal voorspeld als \textbf{exponential average} van de gemeten lengtes van de vorige CPU bursts
\\ SJF =$>$ preemptive of nonpreemptive
\\=$>$ keuze ontstaat wanneer nieuw proces arriveert bij ready queue terwijl vorig proces nog steeds aan het uitvoeren is =$>$ volgende CPU burst van nieuwgearriveerde proces kan korter zijn dan wat er overblijft van huidig uitvoerend proces
\\=$>$ preemptive SJF =$>$ huidig running process be\"eindigen 
\\=$>$ nonpreemptive SJF =$>$ huidig running proces mag CPU burst afmaken
\\=$>$ preemptive SJF wordt soms \textbf{shortest-remaining-time-first} scheduling genoemd

\subsection{Priority Scheduling}
SJF = speciaal geval van \textbf{priority-scheduling} algoritme
\\met elk proces =$>$ priority geassocieerd en dan wordt CPU toegekend aan proces met hoogste priority
\\dezelfde priority processen =$>$ FCFS volgorde 
\\hoge en lage priorities
\begin{itemize}
\item intern gedefineerde priorities
\\=$>$ gebruiken meetbare hoeveelheden om priority van proces te berekenen
\item extern gedefineerde priorities
\\ = ingesteld door criteria buiten operating system
\end{itemize}
priority scheduling =$>$ preemptive of nonpreemptive
\begin{itemize}
\item preemptive
\\ =$>$ preempt CPU als de priority van nieuwgearriveerde proces hoger is dan de priority van huidige running proces
\item nonpreemptive
\\ =$>$ nieuwgearriveerde proces wordt gewoon aan head van de queue gezet
\end{itemize}
groot probleem = \textbf{indefinite blocking} of \textbf{starvation} 
\\=$>$ proces dat klaar is om te runnen maar wachtend op CPU = blocked
\\bij priority scheduling =$>$ sommige low-priority processen moeten indefinitely wachten = indefinite blocking of starvation
\\oplossing voor indefinite blocking van low-priority processen = \textbf{aging}(= gradually priority van processen dat in het systeem lange tijd aan het wachten zijn verhogen)

\subsection{Round-Robin Scheduling}
\textbf{round-robin(RR)} scheduling algoritme = speciaal ontworpen voor time-sharing systemen
\\= gelijkaardig met FCFS scheduling maar preemption is toegevoegd het systeem in staat te stellen om te kunnen switchen tussen processen
\\kleine unit tijd is gedefineerd = \textbf{time quantum} of \textbf{time slice}
\\=$>$ meestal van 10 tot 100 milliseconden in lengte
\\=$>$ ready queue behandeld als circular queue(= CPU scheduler gaat rond in ready queue en wijst CPU toe aan elk proces voor tijdinterval tot 1 quantum)
\\=$>$ ready queue = FIFO queue
\\CPU scheduler haalt eerste proces uit ready queue =$>$ zet timer om te onderbreken na 1 tijd quantum =$>$ dispatches proces:
\begin{itemize}
\item CPU burst is minder dan 1 tijdquantum
\\=$>$ proces laat CPU vrijwillig los
\item CPU burst van huidig running proces is langer dan 1 tijdquantum
\\timer gaat af =$>$ veroorzaakt interrupt bij operating system =$>$ context switch =$>$ proces aan tail ready queue =$>$ CPU volgende proces in ready queue uitvoeren
\end{itemize}

RR =$>$ gemiddelde wachttijd is vaak lang
\\tijdquantum =$>$ lang tegenover context switch time
\\turnaround time hangt ook af van tijd quantum
\\maar: tijdquantum ook niet te lang(anders: RR degenerate in FCFS)
\\=$>$ 80 procent van CPU bursts moeten korter zijn dan time quantum

\subsection{Multilevel Queue Scheduling}
andere klasse schedulingalgoritmes voor situaties waarin processen gemakkelijk geclassificeerd worden in verschillende groepen
\\vaak verdeelt in:
\begin{itemize}
\item \textbf{foreground} (interactive) processes
\item \textbf{background} (batch) processes
\end{itemize}
\chapter{Synchronization}
\chapter{Deadlocks}
in multiprogramming environment =$>$ verschillende processen concurreren voor eindig aantal resources
\\process =$>$ vraagt resources aan
\\=$>$ resources dan niet beschikbaar =$>$ proces in waiting state
\\\textbf{deadlock} = wanneer waiting proces nooit meer van staat kan veranderen doordat resources dat het aangevraagd had  door andere processen vastgehouden worden
\section{System Model}
systeem =$>$ bestaat uit eindig aantal resources om te verdelen over aantal concurrerende processen
\\resources =$>$ verdeeld over types of klassen =$>$ bestaande uit aantal identieke instances
\\proces =$>$ resource aanvragen voor gebruik en loslaten na gebruik
\\=$>$ zoveel resources aanvragen als nodig om taak te kunnen uitvoeren
\\resource gebruiken in volgorde:
\begin{enumerate}
\item request
\\= proces vraagt resource aan 
\\als aanvraag niet onmiddelijk toegekend kan worden =$>$ vragende proces moet wachten tot resource kan verkrijgen
\item use
\\= proces kan gebruik maken van resource
\item release
\\= proces laat resource vrij
\end{enumerate}
set processen =$>$ in deadlocked staat 
\\= elk proces in de set wacht op event dat alleen veroorzaakt kan worden door ander proces in de set
\\resources =$>$ zowel physical resources(bv.: printers, memory space..) als logical resources(bv.: semaphores, files..)
\\deadlock kan ook bij verschillende resource types

\section{Deadlock Characterization}
in deadlock: processen die nooit uitvoering be\"eindigen en system resources opgebruikt waardoor andere jobs voorkomen worden van beginnen
\subsection{Necessary Conditions}
deadlock als volgende condities tegelijk voorkomen
\begin{enumerate}
\item mutual exclusion
\\minstens \'e\'en proces in nonsharable mode
\\= \'e\'en proces tegelijk kan resource gebruiken
\\als ander proces resource aanvraagd wordt dat uitgesteld tot de resource vrijgelaten wordt
\item hold and wait
\\proces moet minstens \'e\'en resource hebben en wachten op bijkomende resources die momenteel bijgehouden worden door andere processen
\item no preemption
\\resources worden niet gepreempt =$>$ resource kan alleen vrijwillig vrijgelaten worden door proces dat het heeft nadat taak uitgevoerd is
\item circular wait
\\zodat $P_{0}$ wacht op resource dat $P_{1}$ heeft, $P_{1}$ wacht op resource van $P_{2}$,..., $P_{n-1}$ wacht op resource van $P_{n}$ en $P_{n}$ op resource $P_{0}$

\end{enumerate} 
vier condities =$>$ niet helemaal onafhankelijk (circular wait impliceert hold-and-wait)

\subsection{Resource-Allocation Graph}
=$>$ meer precies beschreven in termen van gerichte graaf = \textbf{system resource-allocation graph}
\\= bestaat uit set knopen V en set bogen E
\\knopen V:
\begin{itemize}
\item actieve processen
\\= $P = {P_{1},P_{2},...,P_{n}}$
\\= set bestaande uit alle actieve processen in systeem
\\= voorgesteld door cirkel
\item resource types
\\= $R = {R_{1},R_{2},...,R_{m}}$
\\= set bestaande uit alle resource types in systeem
\\= voorgesteld door rechthoek

\end{itemize}
\textbf{request edge} = gerichte boog $P_{i} \rightarrow R_{j}$
\\\textbf{assignment edge}= gerichte boog $R_{j} \rightarrow P_{i}$
\\
\\als graaf geen cycles bevat =$>$ geen proces in systeem is deadlocked
\\als elk resource type precies \'e\'en instance heeft =$>$ cycle impliceert deadlock
\\als elk resource type verschillende instances heeft =$>$ cycle impliceert niet noodzakelijk deadlock

\section{Methods for Handling Deadlocks}
deadlocks afhandelen op 3 manieren:
\begin{itemize}
\item protocol voor voorkomen of vermijden van deadlocks om zo te verzekeren dat systeem nooit in deadlocked state komt
\item toelaten dat systeem in deadlock komt, het detecteren en herstellen
\item probleem negeren en doen alsof deadlocks nooit voorkomen
\end{itemize}
meeste operating system =$>$ 3de oplossing =$>$ application developer moet programma's schrijven die deadlocks afhandelen
\\om te verzekeren dat deadlocks nooit voorkomen:
\begin{itemize}
\item \textbf{deadlock prevention} 
\\= voorziet set methodes zodat minstens \'e\'en van de noodzakelijke condities niet geldt
\\=$>$ beperken hoe resources aangevraagd kunnen worden
\item \textbf{deadlock avoidance}
\\= vereist dat operating system op voorhand bijkomende informatie krijgt met welke resources proces gaat aanvragen en gebruiken tijdens lifetime
\end{itemize}
systeem =$>$ geen deadlock-prevention of deadlock-avoidance =$>$ deadlock kan voorkomen =$>$ tenslotte
\chapter{Memory-Management Strategies}
\chapter{Virtual-Memory Management}
\chapter{File System}
\section{File Concept}
operating system voorziet uniform logical view van bewaarde informatie
\\=$>$ van fysieke eigenschappen van storage devices naar logical storage unit =\textbf{file}(= worden gemapd door operating system op fysieke devices)
\\=$>$ gewoonlijk nonvolatile secondary storage
\\=$>$ voor gebruiker kleinste toekenning van logical secondary storage
\\=$>$ informatie in file = gedefine\"erd door creator
\\=$>$ heeft gedefineerde structuur dat afhangt van zijn type
\begin{itemize}
\item \textbf{text file}
\\= volgorde karakters georganiseerd in regels (en mogelijk pagina's)
\item \textbf{source file}
\\= volgorde functies die verder bestaan uit declaraties gevolgd door uitvoerbare statements
\item \textbf{executable file}
\\= serie code sections dat de lader in het geheugen kan brengen en uitvoeren
\end{itemize}
\subsection{File Attributes}

\chapter{Implementing File-Systems}
\chapter{Mass-Storage Structure}
\chapter{I/O Systems}
\chapter{System Protection}
protection = intern probleem
\chapter{System Security}
\textbf{security}: computer resources =$>$ beschermd worden tegen unauthorized access, kwaardaardige vernietiging of alternation(= afwisseling) en toevalling inconsistency invoegen

\section{The Security Problem}
systeem = \textbf{secure} als de resources gebruikt worden en toegankelijk zijn zoals bedoelt onder alle omstandigheden
\\=$>$ totale security kan niet bereikt worden
\\=$>$ security violations: intentional(malicious) of accidental
\\\textbf{intruder} of \textbf{cracker} = diegenen die security proberen te schenden
\\\textbf{threat} = mogelijkheid voor een security violation(bv.: ondekking van zwakke plek)
\\\textbf{attack} = poging om security te schenden
\\verschillende vormen van accidental en malicious security violations:
\begin{itemize}
\item breach of confidentiality
\\= ongeautoriseerd lezen van data (of diefstal van informatie)
\item breach of integrity
\\= ongeautoriseerd aanpassen van data
\item breach of availability
\\= ongeautoriseerde vernietiging van data
\item theft of service
\\ = ongeautoriseerd gebruik van resources
\item denial of service
\\= voorkomen van rechtmatig gebruik van het systeem
\\\textbf{Denial-of-service(DOS)} = soms accidental
\end{itemize}
attackers gebruiken verschillende standaard methodes om een securityinbreuk te bekomen:
\begin{itemize}
\item \textbf{masquerading}
\\= \'e\'en deelnemer in een communicatie doet alsof hij iemand ander is(andere host of andere persoon)
\\= schending van \textbf{authentication}(= de correctheid van identificatie)
\\=$>$ om privileges te verkrijgen
\item \textbf{replay attack}
\\= malicious of frauduleuze herhaling van een geldige data transmission
\\=$>$ soms alleen: bv. herhalen van overschrijvingsaanvragen voor geld
\\=$>$ meestal samen met \textbf{message modification} om privileges te verkrijgen
\item \textbf{man-in-the-middle-attack}
\\= attacker zit in de data flow van een communicatie en doet zich voor als de zender tegenover de ontvanger en als de ontvanger tegenover de zender
\\=$>$ in network communicatie kan het voorafgegaan worden door \textbf{session hijacking}(waarin een actieve communicatie sessie onderschept wordt)
\end{itemize}
om systeem te beschermen =$>$ veiligheidsmaatregelen op 4 niveau's:
\begin{enumerate}
\item physical
\\= plaats waar computersystemen bewaard worden moeten fysiek beveiligd worden tegen gewapende of heimelijke toegang door intruders (zowel machine rooms als terminals en workstation met toegang tot de machines)
\item{human}
\\authorization =$>$ alleen appropriate user hebben toegang tot systeem
\\zelfs authorized users =$>$ door anderen hun toegang laten gebruiken door bv. omkoping
\\ ook op slinkse wijze aan toegang geraken door \textbf{social engineering}:
\begin{itemize}
\item \textbf{phishing}
\\= een echtlijkende e-mail of webpagina misleidt een gebruiker om vertrouwelijke informatie in te geven
\item \textbf{dumpster diving}
\\= algemene term voor pogingen om informatie te verzamelen om ungeautoriseerde toegang tot de computer te verkrijgen (tussen vuilnis kijken, notities vinden die paswoorden bevatten..)
\end{itemize}
=$>$ deze veiligheidsproblemen = management-en personeelproblemen, behoren niet tot problemen van operating system
\item operating system
\\zichzelf beschermen van accidentele of doelbewuste schendingen van security
\item network
\\=$>$ meeste data in moderne computersystemen =$>$ via private leased line, shared lines(bv.:Internet), wireless connections of dial-up lines
\end{enumerate}
zwakheid op een high level van security(physical of human) laat omzeiling van strenge low-level(operating-system) maatregelen toe

\section{Program Threats}
program threats =$>$ gebruiken breakdown in protection mechanisms van een systeem om programma's aan te vallen
\\processen zijn samen met de kernel de enige manieren om werk uit te voeren op een computer
\\=$>$ doel van crackers: programma schrijven dat breach of security cre\"eert of dat veroorzaakt een normaal proces om zijn gedrag te veranderen en een breach te cre\"eren
\\=$>$ \textbf{backdoor} daemon achterlaten dat informatie voorziet of gemakkelijke toegang toelaat 
\\meest voorkomende methodes waardoor een programma security breaches veroorzaakt:

\subsection{Trojan Horse}
\textbf{Trojan Horse} = code segment dat zijn omgeving misbruikt
\begin{itemize}
\item bij long search paths 
\\=$>$ in andere directory geraken en alles kunnen doen wat de gebruiker doet
\item login programma imitatie
\\=$>$ gebruiker krijgt melding verkeerd wachtwoord en logt daarna pas in op het echte login programma
\\=$>$ authentication key en wachtwoord zijn gestolen door login emulator
\item spyware
\\=$>$ komt mee met download
\\=$>$ doel: downloaden van ads om te tonen op user system, pop-up browsers cre\"eren bij bezoeken van bepaalde sides of informatie verzamelen van user system en het naar een centrale site sturen(= \textbf{covert channels}, waarin heimelijke communicatie gebeurt)
\\spyware = microvoorbeeld van macroprobleem: violation van het principe van minste privilege

\end{itemize}
daemons ge\"installeerd door 2 fouten:
\begin{itemize}
\item user met meer privileges dan nodig (bv. als administrator)
\item operating system by default meer privileges geeft dan een normale user nodig heeft
\end{itemize}
\subsection{Trap Door}
\textbf{trapdoor}= designer van programma of systeem laat een hole achter in de software dat alleen hij kan gebruiken
\\slimme trapdoor =$>$ in compiler
\\trapdoors = moeilijk probleem want om ze te detecteren moet alle source code voor alle componenten van een systeem geanalyseerd worden(kan uit miljoenen lijnen code bestaan)

\subsection{Logic Bomb}
\textbf{logic bomb} = programma dat alleen wanneer een voorgedefineerde set parameters voldaan is een security hole cre\"eert
\\=$>$ hard te detecteren omdat er in normale omstandigheden geen security hole is

\subsection{Stack and Buffer Overflow}
stack-of buffer-overflow attack = meest voorkomende manier voor een attacker buiten het systeem op een netwerk of dial-up connection om ongeautoriseerde toegang te bekomen tot het doelsysteem
\\in essentie buit de attack een bug in een programma uit
\\attacker kan zwakke plek vinden door trail-and-error of het bestuderen van de source code indien beschikbaar
\\ =$>$ programma schrijven voor het volgende:
\begin{enumerate}
\item overflow input field, command-line argument of input buffer tot het naar de stack schrijft
\item overschrijf het huidige returnadres op de stack met het adres van de exploit code geladen in stap 3
\item simple set code schrijven voor de volgende plaats in de stack dat de commands omvat die de attacker wilt uitvoeren
\end{enumerate}
resultaat =$>$ root shell of ander privileged command execution
\\stack-of buffer-overflow kan voorkomen worden met bounds checking
\\mogelijke veiligheidszwakheden van een buffer overflow:
\\=$>$ wanneer functie opgeroepen wordt in typische computerarchitectuur =$>$ doorgegeven parameters, locale variabelen(= soms ook \textbf{automatic variables}),saved frame pointer en returnadres =$>$ in stack frame
\\saved frame pointer =$>$ waarde van stack pointer kan veranderen tijdens functieoproep 
\\=$>$ laat relatieve toegang toe tot parameters en automatische variabelen
\\
\\\'e\'en cracker kan bug bepalen en exploit schrijjven =$>$ iedereen met basis computer skills en toegang tot de exploit(=\textbf{script kiddie}) =$>$ proberen om attack te lanceren op target systems
\\
\\buffer-overflow attack = vooral schadelijk omdat het tussen systemen gerund kan worden en zich verplaatsen over toegelaten communication channels =$>$ moeilijk om te detecteren en te voorkomen (zelfs voorbij firewalls)
\\
\\oplossingen: 
\begin{itemize}
\item CPU =$>$ feature dat de uitvoering van code in stack section van memory niet toelaat 
\\ =$>$ bv.: Sun's SPARC chip en recente versies van Solaris
\item NXfeature
\\= hardware implementatie met gebruik van een nieuwe bit(die associated page als nonexecutable markeert) in de page tables van de CPU's
\end{itemize}

\subsection{Viruses}
\textbf{virus} = fragment van code in een echt programma
\\virussen = self-replicating en ontworpen om andere programma's te infecteren
\\=$>$ richten verwoesting aan in systeem door files aan te passen en te verwijderen en systeemcrashes en programmastoringen te veroorzaken
\\=$>$ zoals de meeste binnendringaanvallen =$>$ zeer specifiek voor architectures, operating systems en applications
\\=$>$ vooral probleem voor gebruikers van PC's
\\ =$>$ UNIX en andere multiuser operating systems = niet vatbaar voor virussen =$>$ uitvoerbare programma's zijn beschermd van schrijven door het operating system (ook andere aspecten van het systeem zijn beschermd)
\\
\\verspreiden zich via e-mail(spam), downloaden besmette programma's van Internet file-sharing services of  uitwisselen van besmette disks
\\=$>$ ook via macros in Microsoft Office files (worden automatisch office ruimte uitgevoerd(Word, Powerpoint, Excel) en zichzelf e-mailen naar anderen in de contactenlijst van gebruiker
\\
\\Hoe werken virussen?
\\wanneer virus target machine bereikt =$>$ \textbf{virus dropper}(meestal Trojan horse) plaatst virus in systeem
\\=$>$ zijn onder te verdelen in verschillende hoofdcategorie\"en(veel behoren tot meer dan \'e\'en categorie)
\begin{itemize}
\item file
\\= infecteert systeem door zichzelf vast te maken aan file
\\= parasitic viruses =$>$ laten geen hele file achter maar hostprogramma functioneert nog
\item boot 
\\= infecteert de boot sector van het systeem door elke uit te voeren wanneer het systeem geboot wordt en voor het operating system geladen is
\\= memory viruses =$>$ komen niet voor in systeem
\item macro 
\\ =$>$ meeste viruses =$>$ geschreven in low-level language (bv.: C, assembly)
\\ =$>$ macrovirussen =$>$ geschreven in high-level language (bv.: Visual Basic)
\\ = worden getriggerd wanneer programma in staat om de macro uit te voeren gerunt wordt
\item source code
\\= zoekt naar source code en verandert het zodat het het virus bevat en het virus helpt verspreiden
\item polymorphic
\\= verandert elke keer de virus' signature wanneer het ge\"installeerd wordt om detectie door antivirus software te vermijden
\\ \textbf{virus signature} = patroon dat gebruikt kan worden om een virus te identificeren (meestal serie bytes)
\item encrypted
\\= omvat decryption code samen met het encrypted virus om detectie te vermijden
\\=$>$ virus decrypts eerst en voert dan uit
\item stealth
\\ = probeert detectie te vermijden door delen van het systeem aan te passen die zouden gebruikt kunnen om het te detecteren
\item tunneling
\\ = probeert detectie van antivirus scanner te omzeilen door zichzelf te installeren in interrupt-handler chain
\\ =$>$ ook in device drivers
\item multipartite
\\= in staat om meerdere delen van systeem te infecteren(boot sectors, memory en files)
\\=$>$ moelijk om te detecteren en contain
\item armored
\\= gecodeerd om moeilijk te maken voor antivirus researchers om te ontrafelen en te begrijpen
\\=$>$ kan ook gecomprimeerd zijn om detectie en disinfectie te vermijden
\end{itemize}
browser virus =$>$ installeert verschillende back-door programma's waaronder \textbf{keystroke logger}(= houdt alles wat op het toetsenbord ingegeven wordt bij, bv.: wachtwoorden)
\\=$>$ ook om ongelimiteerde toegang op afstand door indringer toe te staan en nog \'e\'en om indringer toe te laten spam te sturen met de geïnfecteerde desktop computer
\\
\\\textbf{monoculture} = waarin vele systemen dezelfde harware, operating system en application software runnen
\\=$>$ bestaat zoiets? en verhoogt het de dreiging en schade veroorzaakt door viruses en andere security intrusions?

\section{System and Network Threats}
system en network threats =$>$ misbruiken services en network connections, cre\"eren situatie waarin operating-system resources en user files misbruikt worden
\\soms wordt system en network attack gebruikt om program attack te lanceren en omgekeerd
\\
\\hoe opener operating system is(= meer services enabled en meer functies toegestaan)=$>$ hoe waarschijnlijker dat er bug beschikbaar is om te exploiten
\\=$>$ operating systems streven ernaar om \textbf{secure by default} te zijn
\\=$>$ bijna alle services zijn disabled bij installatietijd en moeten specifiek enabled worden door system administrators
\\=$>$ verminderen \textbf{attack surface}(= aantal manieren waarop een attacker kan proberen in te dringen in het systeem) van systeem
\\
\\ook masquerading en replay attacks worden vaak gelanceerd op netwerken tussen systemen

\subsection{Worms}
\textbf{worm} = process dat \textbf{spawn} mechanism gebruikt om zichzelf te vermenigvuldigen
\\=$>$ spawns kopies van zichzelf, system resources opgebruikend en wellicht alle andere processen uitsluitend
\\=$>$ kunnen netwerk plat leggen
\\worm bestaat uit: \textbf{grappling hook} (= \textbf{bootstrap} = \textbf{vector} programma en main programma
\\=$>$ eens grappling hook op aangevallen computer systeem =$>$ verbinden met de oorspronkelijke machine en kopie van de main worm uploaden naar "hooked" system
\\voorbeelden: Morris en Sobig F.

\subsection{Port Scanning}
port scanning = geen attack, maar manier voor cracker om systeemzwakhedn te detecteren om aan te vallen
\\= typisch geautomatiseerd en bevat een tool dat TCP/IP connectie naar een specifieke port of range van ports probeert te maken 
\\=$>$ indien er tool zou bestaan waarin elke bug van elke service van elk operating gecodeerd zat
\\=$>$ proberen met elke port van sytem(en) te verbinden
\\=$>$ elke service die antwoord =$>$ elke gekende bug proberen
\\=$>$ tool bestaat niet
\\
\\port scans = detecteerbaar =$>$ vaak gelanceerd van \textbf{zombie systems}(= voordien aangetaste onafhankelijke systemen die hun gebruikers bedienen terwijl ze gebruikt worden voor schandelijke doeleinden(bv.: denial-of-service attacks en spam relay))
\\=$>$ source van de attack en de persoon die die gelanceerd heeft is moeilijk te achterhalen
\\=$>$ ook onbelangrijke systemen beveiligen, niet alleen systemen met waardevolle informatie of services

\subsection{Denial of Service}
denial-of-service attacks =$>$ verstoren van legitimate use van systeem
\\=$>$ meestal dringt attacker niet binnen
\\=$>$ gemakkelijker dan inbreken in machine
\\=$>$ meestal network based, 2 categorie\"en:
\begin{itemize}
\item zoveel resources gebruiken dat er geen nuttig werk meer gedaan kan worden
\item network verstoren van facility
\\=$>$ attacks resulteren van misbruik van sommige van de fundamentele functionality van TCP/IP
\\bv.: TCP aanvraag nooit beantwoorden met bevestiging dat de connectie compleet is
\end{itemize}
in algemeen: onmogelijk om DOS attacks te voorkomen, want gebruiken dezelfde mechanisms als normale operatie
\\nog moeilijker te voorkomen en op te lossen: \textbf{distributed denial-of-service(DDOS)} attacks
\\=$>$ worden gelanceerd van meerdere sites tegelijk naar een gemeenschappelijk doel meestal door zombies
\\=$>$ soms geassocieerd met blackmail pogingen

\section{Cryptography as a Security Tool}
\textbf{cryptography} wordt gebruikt om het mogelijke aantal zender en/of ontvangers van boodschap te beperken
\\moderne cryptography =$>$ secrets(=\textbf{keys}) die selectief verdeeld worden naar computer in netwerk en gebruikt worden om boodschappen te verwerken
\\=$>$ ontvanger kan nagaan dat de boodschap gecre\"eerd was door computer die zekere key bezit
\\=$>$ zender kan boodschap coderen zodat alleen een computer met zeker key de boodschap kan decoderen

\subsection{Encryption}
\textbf{encryption} lost een grote verscheidenheid aan communicatieveiligheidsproblemen op =$>$ vaak gebruikt
\\=$>$ beperkt het aantal mogelijke ontvangers van boodschap
\\encryption algorithm bestaat uit volgende componenten:
\begin{itemize}
\item set K van keys
\item set M van messages
\item set C van ciphertexts
\item encryption function: E: K -$>$ (M -$>$ C)
\\= functie voor genereren ciphertexts uit boodschappen
\item decryption function: D: K -$>$ (C -$>$ M)
\\= functie voor genereren boodschappen uit ciphertexts
\end{itemize}

\subsubsection{Symmetric Encryption}
\textbf{symmetric encryption} =$>$ zelfde key wordt gebruikt om te encrypteren en te decrypteren
\\meest gebruikte symmetric encryption algoritme in US voor burgerlijke toepassingen = \textbf{data-encryption standard(DES)} cipher ontworpen door National Institute of Standard and Technology(NIST)
\\=$>$ werkt door 64-bit value en 56-bit key te nemen en een serie transformaties uitvoeren die gebaseerd zijn op vervanging en permutatie operaties 
\\=$>$ DES werkt op block bits per tijd = \textbf{block cipher} en transformaties zijn van block ciphers
\\=$>$ als onveilig beschouwd omdat de keys onuitputtelijk doorzocht kunnen worden met gemiddelde computing resoures
\\=$>$ NIST maakte aanpassing = \textbf{triple DES} waarin DES algoritme 3 keer wordt herhaald (2 encrypties en 1 decryptie) op dezelfde tekst gebruikmakend van 2 of 3 keys
\\in 2001: NIST =$>$ nieuwe block cipher = \textbf{advanced encryption standard(AES)} 
\\=$>$ kan key lengths van 182,192 of 256 bits gebruiken en werkt op 128-bit blocks
\\=$>$ behandelen niet direct boodschappen langer dan hun vereiste block sizes
\\
\\veel \textbf{modes of encryption} gebaseerd op stream ciphers =$>$ gebruikt om veilig langere boodschappen te encrypten
\\=$>$ meest voorkomende \textbf{stream cipher}(= ontworpen om eerder stream bytes of bits te encrypteren en decrypteren dan een block): \textbf{RC4} 
\\=$>$ nuttig wanneer de lengte van communicatie block cipher te traag zou maken
\\=$>$ key is input in pseudo-random-bit generator(= algoritme dat probeert random bits te produceren)
\\=$>$ output is \textbf{keystream} = oneindige set bits die gebruikt kunnen worden om een plaintext stream te encrypteren door gewoon XORing met de plaintext
\\=$>$ XOR(= "eXclusive OR") = operatie die 2 input bits vergelijkt en 1 output bit genereert =$>$ bits hetzelfde =$>$ resultaat = 0, bits verschillend =$>$ resultaat = 1

\subsubsection{Asymmetric Encryption}
\textbf{asymmetric encryption algorithm} = verschillende encryption en decryption keys
\\\textbf{public-key encryption} =$>$ elke zender kan met public key communicatie encrypten maar alleen de key creator kan communicatie decrypten
\\\textbf{RSA} = meest gebruikte asymmetric encryption algoritme
\\=$>$ in RSA: $k_{e}$ = public key, $k_{d}$ = private key
\\gebruik assymetric encryption:
\begin{enumerate}
\item publicatie van de public key
\\= simpel als electronische kopie van key doorgeven of complexer
\item private key (= secret key) =$>$ veilig bewaren aangezien iedereen met deze key boodschappen dan decrypten gecre\"eerd door de matching public key 
\end{enumerate}

\subsubsection{Authentication}
\textbf{authentication} =$>$ beperkt het aantal potenti\"ele zenders van boodschap
\\=$>$ complementair met encryption
\\encryption algorithm bestaat uit volgende componenten:
\begin{itemize}
\item set K van keys
\item set M van messages
\item set A van authenticators
\item function S: K -$>$ (M -$>$ A)
\\= functie voor genereren authenticators uit boodschappen
\item function V: K -$>$ ($M \times A$ -$>$ \{true, false\})
\\= functie voor verifi\"eren van authenticators op boodschappen
\end{itemize}
\textbf{hash function} H(m) cre\"eert een kleine fixed-sized block data = \textbf{message digest} of \textbf{hash value} van message m
\\=$>$ werken door: boodschap te nemen, in blocks opsplisten en blocks verwerken om n-bit hash te verkrijgen
\\=$>$ moet collision resistant zijn
\\veel voorkomende message-digest functions bevatten MD5(nu onveilig beschouwd, produceert 128-bit hash) en SHA-1(160-bit hash)
\\=$>$ message digests = nuttig voor detecteren veranderde boodschappen maar niet zo nuttig als authenticators =$>$ authenticate H(m)
\\2 main types authentication algoritmes:
\begin{itemize}
\item \textbf{message-authentication code(MAC)}
\\= cryptographic checksum wordt gegenereerd van de boodschap door geheime key te gebruiken
\item \textbf{digital-signature algorithm}
\\authenticators = \textbf{digital signatures}

\end{itemize}
encryption en authentication kunnen samen of apart gebruikt worden
\\\textbf{nonrepudation}= levert bewijs dat entity een actie heeft uitgevoerd
\\=$>$ met behulp van digital signatures

\subsubsection{Key Distribution}
bezorging van symmetric key =$>$ grote uitdaging want beide partijen(en niemand anders) hebben de key nodig
\\soms \textbf{out-of-band} = via papier of conversatie =$>$ gaat niet goed op grote schaal
\\= redenen voor pogingen om asymmetric key algoritmes te cre\"eren
\\ maar 1 private key nodig voor alle public keys =$>$ public keys moeten niet beveiligd worden
\\ =$>$ bewaard in \textbf{key ring}
\\ =$>$ bewijs nodig voor wie de eigenaar is van de public key
\\ oplossing = \textbf{digital certificate}= public key digitaal gesigneerd door trusted party
\\=$>$ trusted party ontvangt identificatiebewijs van entity en verklaart dat de public key tot die entity behoort
\\\textbf{certificate authorities} hebben eigen public keys in web browsers voordat ze verdeeld worden =$>$ kunnen getuigen voor andere authorities(digitaal public keys van andere authorities signeren) =$>$ web van vertrouwen

\subsection{Implementation of Cryptography}
network protocols =$>$ georganiseerd in \textbf{layers} met elke layer werkend als de klant van diegene eronder
\\cryptography kan ingebracht worden op bijna elke layer in OSI model
\begin{itemize}
\item IPSec 
\\ = network-layer security, definieert IP packet formats die de toevoeging van authenticators en de encryption van packet inhouden toelaat
\\=$>$ gebruikt symmetric encryption en \textbf{Internet Key Exchange(IKE)}protocol(gebaseerd op public-key encryption) voor key uitwisseling
\\=$>$ wordt veel gebruikt als de basis voor \textbf{virtual private networks(VPN's)}(waarin alle traffic tussen twee IPSec eindpunten gecrypteerd is om een private network te maken uit \'e\'en dat anders public is)
\end{itemize}
talrijke protocols zijn ook ontworpen voor gebruik door applicaties (bv.: PGP voor het encrypteren van e-mail) maar om veiligheid te implementeren moeten de applicaties zelf gecodeerd zijn

\section{User Authentication}
groot veiligheidsprobleem voor operating systems = \textbf{user authentication}
\subsection{Passwords}
meest voorkomende manier om user identity te authenticaten  = \textbf{passwords}
\\wanneer user zichzelf identificeert met user ID of account name =$>$ vraag naar password
\\ =$>$ als wachtwoord ingegeven door user matcht met wachtwoord in systeem =$>$ syteem veronderstelt dat eigenaar het account accessed
\\wachtwoorden worden ook vaak gebruikt om objecten in computersysteem the beveiligen in afwezigheid van meer volledige protection schemes
\\=$>$ verschillende wachtwoorden voor verschillende acces rights
\\in praktijk: meeste systemen vereisen maar \'e\'en wachtwoord voor gebruiker om volledige toegangsrechten te bekomen

\subsection{Password Vulnerabilities}
2 voorkomende manieren om wachtwoorden te raden:
\begin{enumerate}
\item intruder kent user of heeft informatie over user
\\=$>$ vaak gebruiken users voor de hand liggende informatie als wachtwoord
\item brute force
\\= enumeration proberen = alle mogelijke combinaties van valid passwords tot wachtwoord wordt gevonden
\\=$>$ vooral korte wachtwoorden zijn gevoelig voor deze methode
\\= minder succesvol bij systemen die langere wachtwoorden toelaten die zowel hoofd- als kleine letters bevatten, samen met getallen en leestekens
\end{enumerate}
passwords kunnen ook ontdekt worden door visual of electronic monitoring
\begin{itemize}
\item \textbf{shoulder surfing} 
\\ = meekijken over de schouder van gebruiker op toetsenbord wanneer de gebruiker het wachtwoord ingeeft 
\item \textbf{sniff}
\\= iedereen met toegang tot netwerk waar computer op zit kan network monitor toevoegen =$>$ om to sniff(= kijken) alle data die getransferd wordt op het network inclusief user ID's en passwords
\\=$>$ oplossing: encrypteren van data 
\end{itemize}
probleem =$>$ wachtwoorden die neergeschreven worden waar het gelezen kan worden of verloren raken (bv.: systemen met hard-to-remember of lange of frequent veranderende wachtwoorden(aantekening of hergebruik))
\\
\\probleem =$>$ illegal transfer
\\= gebruikers die account delen
\\ =$>$ kan resulteren in unauthorized users die toegang tot het account hebben
\\
\\passwords
\begin{itemize}
\item system-generated
\\=$>$ kunnen moeilijk te herinneren zijn =$>$ neerschrijven 
\item user-selected
\\=$>$ gemakkelijk te raden
\end{itemize}
sommige systemen =$>$ checken wachtwoorden op de moeilijkheid van raden en cracken voor accepteren
\\=$>$ ook age passwords =$>$ dwingen users hun passwords op reguliere tijdstippen te veranderen
\\maar: gebruikers kunnen zo switchen tussen 2 wachtwoorden
\\=$>$ oplossing: password history voor elke gebruiker
\\ verschillende varianten op deze simple password schema's kunnen gebruikt worden

\subsection{Securing Passwords}
computersysteem =$>$ wachtwoorden veilig bewaren
\\ op UNIX: wachtwoorden door hash functie gehaald zodat password list niet geheim moet gehouden worden
\\ =$>$ gegeven x kan f(x) berekend worden, maar gegeven f(x) zou het moeilijk of onmogelijk moeten zijn om x te berekenen
\\systeem heeft wel geen controle meer over wachtwoorden
\\=$>$ iedereen met kopie van password file kan snelle hash routines ertegen runnen
\\=$>$ gebruikt bekend hashing algoritme =$>$ cracker zou paswoorden kunnen bijhouden die al eerder gekraakt zijn
\\daarom: "salt"(= bewaard random getal) in het hashing algoritme
\\=$>$ salt value wordt toegevoegd om te verzekeren dat twee dezelfde plaintext wachtwoorden resulteren in verschillende hashwaarden
=$>$ nieuwere versies van UNIX: password file enkel voor superuser
\\ andere zwakheid in UNIX: behandelt alleen de eerste 8 karakters als significant 
\\goede techniek om paswoord te genereren:
eerste letter van elk woord van gemakkelijk te herinneren zin te gebruiken met zowel hoofd-als kleine letter,getal en leestekens
\\ als spaties zijn toegelaten: \textbf{passphrase}

\subsection{One-Time Passwords}
om problemen van password sniffing en shoulder surfing te vermijden 
\\=$>$ set van  \textbf{paired passwords}
\\= wanneer sessie begint =$>$ selecteert systeem random en toont een deel van password paar =$>$ gebruiker moet andere deel geven
\\ gebruiker = \textbf{challenged} en \textbf{respond} met het correcte antwoord voor die challenge
\\
\\aanpak kan gegeneraliseerd worden tot gebruik van algoritme als wachtwoord
\\=$>$ algoritmische wachtwoorden kunnen niet herbruikt worden(gebruiker typt wachtwoord in en geen entity die dat wachtwoord onderschept kan het herbruiken)
\\=$>$ systeem en gebruiker delen symmetric password
\\=$>$ volgende keer andere authenticator 
\\= \textbf{one-time password} system
\\=$>$ worden op verschillende manieren ge\"implementeerd:
\\software zorgt voor H(pw,ch)
\\pw = kan input zijn door gebruiker of gegenereerd door calculator in synchronizatie met computer
\\=$>$ soms pw = \textbf{personal identification number(PIN)}
\\output = one-time password
\\=$>$ one-time pasword dat input vereist door gebruiken =$>$ \textbf{two-factor authentication}(=$>$ vereist "iets dat je hebt" en "iets dat je weet"
\\ andere variatie = gebruiken \textbf{code book} of \textbf{one-time pad}(= lijst van single-use passwords)
\\=$>$ elk wachtwoord op de lijst word \'e\'en keer gebruikt en dan doorstreept of gewist
\\=$>$ code book moet beschermd worden en het helpt dat de code book niet het systeem identificeert waarvoor de codes authenticators zijn

\subsection{Biometrics}
gebruik van biometric maatregelen
\\=$>$ palm-en handreaders om fysieke toegang te beveiligen =$>$ momenteel te groot en te duur voor normale computer authentication
\\=$>$ fingerprintreaders =$>$ software kan vinger scannen en vergelijken met de bewaarde om te kijken ze matchen
\\\textbf{multifactor authentication} =$>$ beter
\\ maar: strenge authentication op zichzelf is niet voldoende om de ID van de gebruiker te garanderen, kan  steeds gekaapt worden als authentication sessie niet encrypted is

\section{Implementing Security Defenses}
veel security solutions
\\ =$>$ \textbf{defense in depth} = meer lagen van defense zijn beter dan minder lagen
\subsection{Security Policy}
eerste stap =$>$ hebben van \textbf{security policy} =$>$ meestal inclusief statement wat beveiligd wordt
\\=$>$ zonder policy = onmogelijk voor gebruikers en administrators om te weten wat toegestaan is, wat vereist is en wat niet toegestaan wordt
\\=$>$ zou \textbf{living document} moeten zijn dat regelmatig gereviewd en ge\"updated wordt om te verzekeren dat het nog steeds toepasselijk is en gevolgd wordt

\subsection{Vulnerability Assessment}
bepalen of security policy correct is ge\"implementeerd =$>$ vulnerability assessment uitvoeren
\\veel verschillende: social engineering, risk assessment, port scans
\\\textbf{risk assessment} 
\\=$>$ probeert een waarde te plakken op de voordelen van de entity en bepaalt de kans dat security incident de entity treft en de waarde vermindert
\\\textbf{penetration test}
\\= waarin entity gescand wordt op gekende vulnerabilities
\\ =$>$ kernactiviteit van meeste vulnerability assessments
\\vulnerability scans:
\\=$>$ meestal uitgevoerd wanneer computergebruik relatief laag is om impact te minimaliseren
\\
\\computers op netwerk =$>$ vatbaarder voor security attacks dan standalone systems
\\=$>$ eerder attacks van groot aantal onbekende access points dan van gekende access points 
\\=$>$ potentieel ernstig veiligheidsprobleem
\\
\\US government =$>$ system is zo veilig als verste connectie
\\vaak onmogelijk om machine in room op te sluiten en geen remote access toe te staan
\\vulnerability scans =$>$ om problemen met network security te vinden
\\=$>$ zoeken in netwerk naar poorten die request beantwoorden
\\als services ingeschakeld zijn die dat niet zouden mogen zijn =$>$ kan toegang ertoe geblokkeerd worden of ze kunnen uitgeschakeld worden
\\scans =$>$ testen poorten op gekende zwakheden 
\\
\\security tools kunnen door crackers misbruikt worden
\\=$>$ sommige mensen pleiten voor \textbf{security through obscurity} (= geen tools schrijven om security te testen omdat ze gebruikt en uitgebuit kunnen worden om security holes te vinden
\\anderzijds: crackers kunnen eigen tools schrijven

\subsection{Intrusion Detection}
\textbf{intrusion detection} =$>$ pogingen of succesvolle inbraken in computersystemen ontdekken en beginnen en gepaste antwoorden geven aan inbraken
\\veel verschillende technieken:
\begin{itemize}
\item tijdstip
\\=$>$ terwijl inbraak bezig is of erna
\item onderzochte types input
\\bv.: user-shell commands, process system calls..
\item omvang antwoordmogelijkheden
\begin{itemize}
\item simpele antwoordvormen
 \\bv.: administrator waarschuwen of de inbraak stoppen
\item geavanceerde antwoordvormen
\\ =$>$ verleggen van intruder's activiteit naar \textbf{honeypot} (= valse resource zichtbaar voor attacker die echt lijkt)
\\=$>$ systeem monitort en wint informatie over attack
\end{itemize}
\end{itemize}
veel vrijheid in ontwerp voor detecting intrusions =$>$ grote omvang oplossingen:
\begin{itemize}
\item \textbf{intrusion-detection systems(IDSs)} 
\\= alarm wanneer intrusion gedetecteerd wordt
\item \textbf{intrusion-prevention systems(IDPs)}
\\= gedragen zich als routers, met passerend verkeer tot intrusion gedetecteerd wordt(blokkeren intrusion en rest van het verkeer)
\end{itemize}
hoe intrusion detecteren?
\begin{itemize}
\item \textbf{signature-based detection} 
\\= systeeminput of netwerkverkeer wordt onderzocht op specifieke behavior patterns (= \textbf{signatures}) gekend om attacks aan te geven
\\=$>$ probeert gevaarlijke behaviors te karakteriseren en te detecteren als er \'e\'en voorkomt
\item \textbf{anomaly detection}
\\= probeert via verschillende technieken abnormaal behavior in computersystemen te ontdekken
\\=$>$ gebaseerd op veronderstelling dat intrusions vaak abnormaal behavior veroorzaken
\\=$>$ probeert normaal of niet-gevaarlijk behaviors te karakteriseren en te detecteren wanneer er iets anders voorkomt dan deze behaviors

\end{itemize}
=$>$ leveren IDSs en IDPs met zeer verschillende eigenschappen op 
\\=$>$ anomaly detection =$>$ vindt vooraf ongekende methods van intrusion = \textbf{zero-day attacks}
\\=$>$ signature-based detection =$>$ alleen gekende attacks identificeren die gecodeerd kunnen worden in herkenbaar patroon
\\=$>$ anomaly detection =$>$ normaal systeembehavior moet accuraat gebenchmarkt worden:
\begin{itemize}
\item intrusion al in system 
\\=$>$ intrusive behavior opgenomen in normale benchmark(= criterium)
\item benchmark =$>$ behoorlijk compleet beeld van normaal gedrag
\\=$>$ anders veel \textbf{false positives}(= valse alarmen) en \textbf{false negatives}(= gemiste intrusions)
\end{itemize}
P|A) = kans dat een alarm op een intrusion wijst =$>$ kan berekend worden met \textbf{Bayes' theorem}
\\ in systemen waar security admintistrator elk alarm onderzoekt =$>$ hoge ratio van valse alarmen (= "Christmas tree effect") =$>$ verspilling en administrator gaat alarmen negeren
\\dus: IDSs en IDPs moeten extreem lage false-alarm rate aanbieden

\subsection{Virus Protection}
=$>$ vaak voorzien door antivirus programma's 
\\sommige programma's =$>$ effectief tegen gekende virussen
\\=$>$ alle programma's op systeem doorzoeken voor specifiek patroon of instructies dat virus opstelt
\\=$>$ gevonden =$>$ verwijderen instructies = \textbf{disinfecting} programma
\\antivirus programma's =$>$ catalogi van duizenden virussen waarop ze zoeken
\\
\\zowel virussen als antivirus software =$>$ meer geavanceerd
\\antivirus programma's zoeken naar families van patronen in plaats van enkel patroon om virus te identificeren
\\=$>$ sommige implementeren verschillende detection algoritmes
\\bv.: decompress compressed virussen, process anomalities
\\ook =$>$ program runnen in \textbf{sandbox} (= gecontroleerde of emulated sectie van systeem, waarin antivirus software gedrag van code analyseer voordat het ongemonitord gerund wordt)
\\
\\beste bescherming tegen computervirussen = preventie of toepassen van \textbf{safe computing}
\\bv.: nieuwe kopiën van software legitiem kopen
\\tegen macrovirussen: uitwisselen van Microsoft Word documents in alternatief file format = \textbf{rich text format(RTF)}
\\=$>$ RTF bevat niet de mogelijkheid om macro's vast te hechten
\\andere verdediging = vermijden van openenen van e-mailbijlagen van onbekende gebruikers
\\andere bescherming =$>$ staat vroegtijdige detectie toe
\\=$>$ helemaal hard disk reformatteren

\subsection{Auditing, Accounting, and Logging}
\begin{itemize}
\item logging
\\= algemeen of specifiek
\\=$>$ alle system-call uitvoeringen kunnen gelogd worden voor analyse van program behavior(of misbehaviour)
\\=$>$ meestal verdacht gebeurtenissen: bv.: authentication failures en authorization failures =$>$ inbraakpogingen
\item accounting
\\= gebruikt om performance changes te vinden die security problemen kunnen onthullen
\end{itemize}
\section{Firewalling to Protect Systems and Networks}
hoe betrouwbare computer veilig kan verbonden worden met onbetrouwbaar netwerk
=$>$ oplossing: \textbf{firewall} = computer, toestel of router dat betrouwbaar en ontbetrouwbare systemen scheidt
\\=$>$ network firewall limiteert network access tussen de twee \textbf{security domains} en monitort en logt alle connecties
\\=$>$ kan ook verbindingen gebaseerd op source of destination address, source of destination poort of de richting van de verbinding limiteren
\\
\begin{itemize}
\item Internet
\\vaak ge\"implementeerd als onbetrouwbaar domein
\item \textbf{demilitarized zone (DMZ)}
\\= semitrusted en semisecure network
\item company's computers
\\=$>$ derde domein
\end{itemize}
verbindingen =$>$ toegestaan van Internet naar DMZ computers en van company computers naar Internet 
\\=$>$ niet toegestaan van Internet of DMZ computers naar company computers
\\
\\firewall zelf =$>$ veilig en attack-proof
\\=$>$ voorkomt geen attacks =$>$ \textbf{tunnel}(= reizen in protocols of verbindingen die de firewall toelaat)
\\andere zwakheid van firewall:
\\ = \textbf{spoofing}
\\=$>$ waarin ongeautoriseerde host doet alsof hij geautoriseerde host is door aan aantal authorization criteria's te voldoen
\\
\\in aanvulling tot meest gebruikte netwerk firewalls =$>$ nieuwe soorten:
\begin{itemize}
\item \textbf{personal firewall}
\\= softwarelaag ofwel in operating system of wel toegevoegd als applicatie
\item \textbf{application proxy firewall}
\\= verstaat protocols van applicaties over netwerk
\item sommige firewalls =$>$ ontworpen voor \'e\'en specifiek protocol
\\bv.: \textbf{XML firewall}
\\=$>$ specifieke doel om XML traffic te analyseren en niet toegelaten of malformed XML te blokkeren
\item \textbf{system-call firewalls}
\\= zitten tussen applicaties en kernel
\\=$>$ monitoren system-call uitvoeringen
\end{itemize}

\section{Computer-Security Classifications}
=$>$ 4 security classificaties in systems door U.S. Department of Defense Trusted Computer Evaluation Criteria: A, B, C en D
\begin{itemize}
\item division D
\\= laagste niveau classificatie
\\= minimale protectie
\\=$>$ gebruikt voor systemen die niet voldoen aan vereisten van alle andere security classes
\item division C
\\= voorziet protectie en verantwoordelijkheid van gebruikers voor acties door gebruik van audit capabilities
\begin{itemize}
\item division C1
\\= bevat controlevormen dat gebruikers toestaan om priv\'e-informatie te beschermen en andere gebruikers weg te houden van per ongeluk lezen of vernietigen van die data
\\= waarin cooperating users toegang tot data hebben op hetzelfde niveau
\\\textbf{trusted computer base(TCB)}
\\= geheel van alle protectiesystemen binnen computersysteem(hardware, software, firmware) dat correct security policy afdwingt
\\ TCB van C1-systeem contoleert toegang tussen gebruikers en files door toe te staan dat gebruiker specifiek objecten deelt met genoemde individuen of gedefinieerde groepen
\\=$>$ vereist user identification
\item division C2
\\= voegt individual-level access control toe aan de vereisten van C1-systeem

\end{itemize}
\item division B
\\ = verplichte protectiesystemen met alle eigenschappen van class-C2 systeem
\\ =$>$ voegen gevoeligheidsniveaulabels toe aan elk object in systeem
\begin{itemize}
\item division B1
\\ = labels die gebruikt worden voor beslissingen met betrekking tot verplichte toegangscontrole
\\=$>$ geeft ook gevoeligheidsniveau weer op de top en bottom van elke pagina van human-readable output
\\=$>$ op zijn minst twee hi\"erarchische niveaus security ondersteunen
\item division B2
\\=$>$ breidt labels uit tot elke system resource zoals storage objects
\\=$>$ fysieke devices krijgen minimale en maximale security levels toegekend dat systeem gebruikt om beperkingen op te leggen gebaseerd op de fysieke omgevingen waarin de devices zich bevinden 
\\=$>$ ondersteunt covert channels en auditing van events die kunnen leiden tot uitbuiting van covert channels
\item division B3
\\ =$>$ acces-control list dat aanduidt of gebruiker of groep al dan niet toegang heeft tot gegeven genoemd object
\\=$>$ mechanisme om events te controleren die inbreuk op security policy kunnen aangeven (waarschuwt security administrator en eventueel event be\"eindigen)
\end{itemize}
\item division A
\\= hoogste niveau classificatie
\\architecturaal is A1 functioneel equivalent met B3 maar door gebruikt van formele ontwerpspecificaties en verificaties zodat er een grote verzekering is dat de TCB correct ge\"implementeerd is

\end{itemize}
gebruik TCB verzekert dat systeem aspecten van security policy kan afdwingen
\\TCB specifieert niet wat policy moet zijn
\\meestal =$>$ gegeven computing environment ontwikkeld security policy voor \textbf{certification}
en heeft het plan \textbf{accredited}(= officieel erkend) door security agency
\\sommige computing environments vereisen andere certification
\\bv.: zoals voorzien door TEMPEST( beschermt tegen elektronisch afluisteren)

\section{An Example: Windows 7}
Microsoft Windows 7 = general-purpose operating system ontworpen om verscheidenheid aan security features en methodes te ondersteunen
\\security model
\\=$>$ gebaseerd op idee van \textbf{user accounts}
\\=$>$ laat creatie van aantal user accounts toe 
\\=$>$ ge\"identificeerd door systeem met unieke security ID
\\
\\Wanneer user inlogt =$>$ creatie van \textbf{security access token} (= omvat security ID voor gebruiker, security ID van eventuele groepen en lijst speciale privileges van user)
\\=$>$ elk proces dat Windows 7 runt voor user =$>$ kopie van access token
\\=$>$ om toegang toe te laten of te weigeren tot object
\\user authentication meestal door user name en wachtwoord
\\
\\gebruikt idee van subject om te verzekeren dat er niet te veel toegang wordt toegekend
\\\textbf{subject} =$>$ gebruikt om toelatingen voor elk programma dat gebruiker runt te volgen en te beheren

\end{document} 